

%todo double check citation RealGIT

The chapter presents several criteria for testing stability notions from Definition~\ref{defn:StabilityGroupTopological}. These criteria are used throughout the thesis. We give corresponding references in each section.

\paragraph{Organization.}
Section~\ref{sec:HilbertMumford} contains the Hilbert-Mumford Criterion for tori, and more generally, for reductive groups. In Section~\ref{sec:KempfNess} we introduce moment maps and moment polytopes, and state the Kempf-Ness Theorem, which is of particular importance for this thesis. Afterwards, we deduce from King's Criterion a characterization for being (semi)stable under the left-right action, Section~\ref{sec:King}. While all previous criteria require a reductive group, Popov's Criterion in Section~\ref{sec:Popov} can be used to test polystability under a solvable group. 

%content: numerical Mumford?, Hilbert-Mumford, Kempf-Ness, moment maps and moment polytopes, convexity theorems;
%Popov criterion!
%King's criterion for quivers and its specialization to the Kronecker quiver


%=========== Hilbert-Mumford ========================

\section{Hilbert-Mumford Criterion} \label{sec:HilbertMumford}

In the following we formulate the Hilbert-Mumford Criterion for tori and then for general reductive groups. Afterwards, we focus on the torus case and provide two detailed proofs. The latter is mainly based on \cite[Appendix~A]{DiscretePaper}.

\medskip

Let $G$ be a complex algebraic group. An \emph{(algebraic) one-parameter subgroup}\index{one-parameter subgroup} (short: 1-psg) of $G$ is a morphism $\lambda \colon \CC^\times \to G$ of complex algebraic groups $G$.

\begin{example}\label{ex:OnePSGsGTd}
	The algebraic one-parameter subgroups of the torus $\GT_d(\CC)$ are in bijection with $\ZZ^d$. The 1-psg given by $(\lambda_1,\ldots,\lambda_d) \in \ZZ^d$ is
	\begin{equation}\label{eq:OnePSG-GTd}
			\lambda \colon \CC^\times \to \GT_d(\CC) , \quad t \mapsto \diag \big( t^{\lambda_1}, \ldots, t^{\lambda_d} \big) .
	\end{equation}
	By abuse of notation, we denote by $\lambda$ both the 1-psg and the vector in $\ZZ^d$.
	\hfill\exSymbol
\end{example}

\begin{theorem}[Hilbert-Mumford for Tori, {\cite[p.~173]{KraftBook}}] \label{thm:generalHilbertMumfordTorus}
	\ \\
	Let $\pi \colon T \to \GL(V)$ be a rational representation of a complex torus $T$. Fix $v \in V$ and let $w \in $ $\overline{T \cdot v} \backslash T \cdot v $. Then there exists an algebraic one-parameter subgroup $\lambda \colon \CC^\times \to T$ such that
		\[ \lim_{t \to 0} \, \lambda(t) \cdot v \in T \cdot w. \]
	In particular, if $v \neq 0$ is $T$-unstable, then choosing $w=0$ gives $\lim_{t \to 0} \, \lambda(t) \cdot v = 0$.
	%Consider the action of $\GT_d(\CC)$ on $\CC^m$ given by matrix $A \in \ZZ^{d \times m}$, and fix $v \in \CC^m$. If $w \in \overline{\GT_d(\CC) \cdot v} \backslash \GT_d(\CC) \cdot v$, then there exists a one-parameter subgroup that scales $v$ to an element of $\GT_d(\CC) \cdot w$ in the limit.
\end{theorem}

We give a proof of the special case of an unstable $v$ and $w=0$ below in Theorem~\ref{thm:specialHilbertMumford}. Furthermore, Theorem~\ref{thm:generalHilbertMumfordTorus} allows for a characterization of all stability notions under a torus, see Theorem~\ref{thm:HMtorusWeightPolytope} below.
For a general reductive group we have the following statement, also see \cite[Theorem~4.2]{birkes1971orbits} (proof due to R. Richardson).

\begin{theorem}[Hilbert-Mumford for Reductive Groups, {\cite[Theorem~6.9]{PopovVinberg}}]
	\label{thm:HilbertMumfordReductive}
	Let $\pi \colon G \to \GL(V)$ be a rational representation of a complex reductive group $G$. Fix $v \in V$ and let $G \cdot w$ be the unique closed orbit\footnote{compare Theorem~\ref{thm:GeneratingInvariantsSeparate}}
	in $\overline{G \cdot v}$. Then there exists an algebraic one-parameter subgroup $\lambda \colon \CC^\times \to G$ of $G$ such that
		\[ \lim_{t \to 0} \, \lambda(t) \cdot v \in G \cdot w. \]
	In particular, if $v \neq 0$ is $G$-unstable, then $w=0$ yields $\lim_{t \to 0} \, \lambda(t) \cdot v = 0$.
\end{theorem}

Hence, the Hilbert-Mumford Criterion ensures that being unstable under the action of a reductive group is always witnessed by a one-parameter subgroup.

\begin{remark} \label{rem:HilbertMumfordReductive}
	Regarding Theorem~\ref{thm:HilbertMumfordReductive} we point out the following.
	\begin{itemize}
		\item[(i)] In contrast to case of tori (Theorem~\ref{thm:generalHilbertMumfordTorus}), for a reductive group $G$ one can in general \emph{not} choose any $G$-orbit in $\overline{G \cdot v} \backslash G \cdot v$. Indeed, Example~1 in \cite[§6.8]{PopovVinberg} shows that the assumption ``$G \cdot w$ is the unique closed orbit in $\overline{G \cdot v}$'' in Theorem~\ref{thm:HilbertMumfordReductive} is necessary.
		
		\item[(ii)] If the whole setting in Theorem~\ref{thm:HilbertMumfordReductive} is defined over $\RR$ and $v \in V_\RR$, then one can choose a one-parameter subgroup that is defined over $\RR$, by a result of Birkes \cite[Theorem~5.2]{birkes1971orbits}. In fact, it was proven by Kempf that such a rationality result of the Hilbert-Mumford Criterion holds for \emph{any} perfect field, \cite[Corollary~4.3]{kempf1978instability}.
		
		\item[(iii)] The Hilbert-Mumford Criterion is an important proof ingredient for the Kempf-Ness Theorem~\ref{thm:KempfNessAKRS} , both over the complex and over the real numbers.
		\hfill\remSymbol
	\end{itemize}
\end{remark}

We will need the following result, that is often shown as an intermediate step to prove Hilbert-Mumford.

\begin{theorem}[{\cite[Theorem~3.25]{Wallach}}] \label{thm:Wallach3-25}
	Let $G \subseteq \GL_N(\CC)$ be Zariski closed and self-adjoint. Set $K := G \cap \Un_N$ and $T:= (G \cap \GT_N(\CC))^\circ$. Consider a rational representation $\pi \colon G \to \GL(V)$ and fix $v \in V$. Let $G \cdot w$ be the unique closed orbit in $\overline{G \cdot v}$. Then there exists $k \in K$ such that $\overline{T \cdot (k \cdot v)} \cap G \cdot w \neq \emptyset$. In particular, if $v$ is $G$-unstable, then $w=0$ and hence $0 \in \overline{T \cdot (k \cdot v)}$.
\end{theorem}


\subsubsection{Proofs in the Torus Case}

We provide a proof of the ``classical'' Hilbert-Mumford Theorem for a torus, and for characterizations via the so-called weight polytope. The proofs are taken from \cite[Appendix~A]{DiscretePaper} and are intended to be accessible to a wide audience. They illustrate that the Hilbert-Mumford Criterion in the torus case is an instance of linear programming duality and its many variants, compare \cite[Chapter~7]{SchrijverBook}. 

Let $T \subseteq \GT_N(\CC)$ be a complex sub-torus and set $T_K := T \cap \Un_N$. Consider a rational representation $\pi \colon T \to \GL(V)$ with set of weights $\Omega(\pi) \subseteq \imag \Lie(T_K) \subseteq \RR^N$ and weight space decomposition $V = \bigoplus_\omega V_\omega$, see Theorem~\ref{thm:WeightSpaceDecomposition}. 
Given $v \in V$, we write $v = \sum_{\omega} v_\omega$ with $v_\omega \in V_\omega$. Define the \emph{support}\index{support} of $v$ with respect to $\pi$ as
\begin{align*}
	\supp(v) := \lbrace \omega \in \Omega(\pi) \mid v_\omega \neq 0 \rbrace.
\end{align*}
Furthermore, the \emph{weight polytope} of $v$ is
\begin{equation}\label{eq:WeightPolytopeDefn}
	\Delta_{T}(v) := \conv \big\{ \omega \mid \omega \in \supp(v) \big\} \subseteq \imag \Lie(T_K) \subseteq \RR^N.
\end{equation}
Using the weight polytope, the Hilbert-Mumford Criterion, Theorem~\ref{thm:generalHilbertMumfordTorus}, actually yields a characterization of all stability notions from Definition~\ref{defn:StabilityGroupTopological}; compare Theorem~\ref{thm:HMtorusWeightPolytope} below. Since any torus is isomorphic to $\GT_d(\CC)$, we restrict for concreteness to this situation.

Let $T = \GT_d(\CC)$ act on $V = \CC^m$ via the matrix $A \in \ZZ^{d \times m}$, see Example~\ref{ex:GeneralGTaction}. The weights of this action are the columns $A_j$ of the matrix $A$ with corresponding weight vector $e_j \in \CC^m$. Therefore, the weight polytope~\eqref{eq:WeightPolytopeDefn} of $v \in \CC^m$ is
	\[ \Delta_A(v) := \Delta_T(v) =  \conv \big\{ A_j \mid v_j \neq 0 \big\} .\]
It is convenient to remember the weight matrix $A$ in this notation.

Now, we head towards proving  the special case of Theorem~\ref{thm:generalHilbertMumfordTorus}. For this, let $\lambda$ be a one-parameter subgroup of $T = \GT_d(\CC)$ as in \eqref{eq:OnePSG-GTd}.
For $v \in \CC^m$, the $j^{th}$ entry of $\lambda(t) \cdot v$ is
\[ 
(\lambda(t) \cdot v)_j = t^{\langle \lambda, A_j \rangle} v_j . 
\] 
We consider $\lim_{t \to 0} \lambda(t) \cdot v$. Its $j^{th}$ entry is zero for $j \notin \supp(v)$. For $j \in \supp(v)$, we have three possibilities
\begin{equation}\label{eq:1psgLimit}
	\left(\lim_{t \to 0} \;\; \lambda(t) \cdot v \right)_j =
	\begin{cases}
		0       & \quad \text{if } \langle \lambda, A_j \rangle > 0\\
		v_j     & \quad \text{if } \langle \lambda, A_j \rangle = 0\\
		\infty  & \quad \text{if } \langle \lambda, A_j \rangle < 0
	\end{cases}
\end{equation}

To prove the Hilbert-Mumford Criterion, we need the following result from the realm of linear programming duality, Farkas' lemma, etc.

\begin{theorem}[Gordan's Transposition Theorem, {\cite[§7.8 Equation~(31)]{SchrijverBook}}] \label{thm:Gordan}
	Let $\FF \in \{\QQ, \RR\}$  and $B \in \FF^{d \times k}$. There is a vector $x \in \FF^k$ with $x \geq 0$, $x \neq 0$ and $Bx = 0$ if and only if there is no vector $y \in \FF^d$ with $y\T B > 0$.
\end{theorem}

The classical statement of the Hilbert-Mumford Criterion for a torus action is as follows, see e.g., \cite[Proposition~5.3]{PopovVinberg} and~\cite[Lemma~3.4]{birkes1971orbits}.

\begin{theorem}\label{thm:specialHilbertMumford}
	Consider the action of $\GT_d(\CC)$ on $\CC^m$ via the matrix $A \in \ZZ^{d \times m}$. Let $v \in \CC^m \backslash \{0\}$ with zero in its orbit closure. Then there exists a one-parameter subgroup $\lambda$ of $\GT_d(\CC)$ such that $\; \lim_{t \to 0} \, \lambda(t) \cdot v = 0$.
\end{theorem}

\begin{proof}[Proof of Theorem~\ref{thm:specialHilbertMumford}]
	The proof follows \cite{Sury}. We have $\supp(v) \neq \emptyset$ as $v \neq 0$. After reordering the entries of $v$, we can assume without loss of generality that $\supp(v) = [k]$ for some $k \leq m$.
	
	We seek a one parameter subgroup $\lambda \colon \CC^\times \to \GT_d(\CC)$ such that $\lim_{t \to 0} \, \lambda(t) \cdot v$ is zero. 
	From the form of a one parameter subgroup in~\eqref{eq:OnePSG-GTd} and the limiting behaviour from~\eqref{eq:1psgLimit}, we see that this is equivalent to showing that
	\begin{equation}
		\label{eqn:gordan}
		\exists \, \lambda \in \ZZ^d \colon \forall\, j \in [k] = \supp(v) \colon \quad  \langle \lambda, A_j \rangle > 0 .
	\end{equation}
	Let $B \in \ZZ^{d \times k}$ be the submatrix consisting of the first $k$ columns of $A = (a_{ij})$. Then \eqref{eqn:gordan} reformulates as: there exists $\lambda \in \ZZ^d$ with $\lambda\T B > 0$. Hence, by Theorem~\ref{thm:Gordan} with $\FF = \QQ$, \eqref{eqn:gordan} is equivalent\footnote{Note that the existence of a $y \in \QQ^d$ with $y\T B > 0$ is, after multiplying with a common denominator, equivalent to the existence of some $\lambda \in \ZZ^d$ with $\lambda\T B >0$. In \cite{Sury} the equivalence of~\eqref{eqn:gordan} and~\eqref{eq:proofClassicalHM} is stated in Lemma~1.1.}
	to the following statement:
	\begin{equation}\label{eq:proofClassicalHM}
		\begin{matrix}    \text{if} \, \, x = (x_1,\ldots,x_k) \in \QQ^k \backslash \{0\} \, \, \text{is such that} \, \,
			a_{i1} x_1 + \cdots + a_{ik} x_k = 0 \, \, \\ \text{for all} \, \, i \in [d], \,\,
			\text{ then at least two entries of $x$ are of opposite sign.}
		\end{matrix} 
	\end{equation}
	Thus, it remains to prove~\eqref{eq:proofClassicalHM}.
	Since $0 \in \overline{\GT_d(\CC) \cdot v}$, there exists a sequence $t^{(n)} = (t^{(n)}_1,\ldots,t^{(n)}_d) \in \GT_d(\CC)$ with $t^{(n)} \cdot v \to 0$ as $n \to \infty$. In coordinates, 
	\begin{equation}\label{eq:lambdaN}
		\forall \, j \in [k] \colon \quad  \big( t^{(n)}_1 \big)^{a_{1j}} \cdots \big( t^{(n)}_d \big)^{a_{dj}} \to 0 \quad \text{ as } \quad n \to \infty .
	\end{equation}
	The hypothesis of~\eqref{eq:proofClassicalHM} is that we have $x \in \QQ^k \backslash \{0\}$ with $x_1 a_{i1} + \cdots + x_k a_{ik} = 0$ for all $i \in [d]$. Without loss of generality, we can assume $x_1$ is non-zero and therefore
	\begin{equation*} %\label{eqn:ai1}
			\forall \, j \in [k] \colon \quad -a_{i1} = \frac{x_2}{x_1} a_{i2} + \cdots + \frac{x_k}{x_1} a_{ik} ,
	\end{equation*}
	which implies
	\begin{equation}\label{eq:classicalHMcontra}
		\prod_{i=1}^d \Big(t^{(n)}_i \Big)^{-a_{i1}} = 
		\left( \prod_{i=1}^d \left(t^{(n)}_i \right)^{a_{i2}} \right)^{\frac{x_2}{x_1}} \cdots
		\left( \prod_{i=1}^d \left(t^{(n)}_i \right)^{a_{ik}} \right)^{\frac{x_k}{x_1}}.
	\end{equation}
	If $x_j / x_1 \geq 0$ for all $j \in \{ 2,\ldots,k \}$, then the right-hand side of \eqref{eq:classicalHMcontra} either equals one (if all $x_j / x_1$ are zero) or tends to zero (if there exists some $j$ with $x_{j} / x_1 > 0$). But the left-hand side of~\eqref{eq:classicalHMcontra} tends to infinity as $n \to \infty$, since it is the inverse of~\eqref{eq:lambdaN} for $j=1$. Hence $x_j / x_1$ must be strictly negative for some $j$, i.e., $x_1$ and $x_j$ have opposite signs.
\end{proof}

We note that the generalization in Theorem~\ref{thm:generalHilbertMumfordTorus} can be proven by similar arguments from polyhedral geometry.

Now, let us turn towards Hilbert-Mumford in terms of the weight polytope. We use the following consequence of Gordan's Theorem~\ref{thm:Gordan}.

\begin{cor}\label{cor:Gordan}
	Let $B \in \ZZ^{d \times k}$ and let $\Delta_B \subseteq \RR^d$ be the polytope spanned by the columns of $B$. Then $0 \notin \Delta_B$ if and only if there exists $\lambda \in \ZZ^{d}$ with $\lambda\T B > 0$.
\end{cor}

\begin{proof}
	First, note that $0 \in \Delta_B$ is equivalent to the existence of $x \in \RR^d \backslash \{0\}$ such that $x \geq 0$ and $Bx = 0$. Thus, if there is $\lambda \in \ZZ^{d}$ with $\lambda\T B > 0$, then $0 \notin \Delta_B$, by Theorem~\ref{thm:Gordan} for $\FF = \RR$. On the other hand, if $0 \notin \Delta_B$ then there is $y \in \RR^d$ with $y\T B > 0$, again by Theorem~\ref{thm:Gordan}. The existence of such a vector $y$ ensures that we can in fact choose $y \in \QQ^d$. After multiplying with a common denominator, we obtain some $\lambda \in \ZZ^{d}$ with $\lambda\T B > 0$.
\end{proof}

Finally, we prove a full characterization of stability notions via the weight polytope. Its formulation is based on \cite[Theorem~3.4]{DiscretePaper} and the proof is taken from \cite[Appendix~A]{DiscretePaper}. Given a polytope $P \subseteq \RR^d$, we denote its \emph{interior}\index{interior of a polytope} by $\interior(P)$ and its \emph{relative interior}\index{relative interior of a polytope}\index{interior of a polytope!relative} by $\relint(P)$.

\begin{theorem}[Hilbert-Mumford Criterion via the Weight Polytope]
	\label{thm:HMtorusWeightPolytope} %formerly thm:HMtorus
	\ \\
	Consider the action of $\GT_d(\CC)$ on $\CC^m$ given by matrix $A \in \ZZ^{d \times m}$. For $v \in \CC^m$, we have
	\[\begin{matrix}
		(a) & v \text{ unstable} & \Leftrightarrow & 0 \notin \Delta_A(v) \\
		(b) & v \text{ semistable} & \Leftrightarrow & 0 \in \Delta_A(v) \\
		(c) & v \text{ polystable} &  \Leftrightarrow & 0 \in \relint(\Delta_A(v)) \\
		(d) & v \text{ stable} & \Leftrightarrow & 0 \in \interior(\Delta_A(v)) \end{matrix}
	\]
	If $\GT_d(\CC)$ acts on $\CC^m$ given by matrix $A \in \ZZ^{d \times m}$ \emph{with linearization} $b \in \ZZ^d$, then the same statements~(a) -- (d) apply when replacing zero by $b$.
\end{theorem}

\begin{remark}
	Of course, Theorem~\ref{thm:HMtorusWeightPolytope} also holds for the setting $T \subseteq \GL_N(\CC)$ with weight polytope $\Delta_T(v)$ as in \eqref{eq:WeightPolytopeDefn}. In that situation, the interior in part~(d) has to be taken with respect to the $\RR$-vector space $\imag \Lie(T_K)$.
	\hfill\remSymbol
\end{remark}

We give a (hopefully) elementary and accessible proof of Theorem~\ref{thm:HMtorusWeightPolytope}. Other references are \cite[Theorem~9.2]{DolgachevBook} and \cite[Theorem~1.5.1]{Szekelyhidi}.

\begin{proof}[Proof of Theorem~\ref{thm:HMtorusWeightPolytope}]
	Set $T := \GT_d(\CC)$.
	We first prove part~(a), and hence (b) as well. If $v=0$, then the polytope $\Delta_A(v)$ is empty, hence $0 \notin \Delta_A(v)$. Assume $v \neq 0$. Then $v$ is unstable if and only if there exists some $\lambda \in \ZZ^d$ such that $\langle \lambda, A_j \rangle > 0$ for all $j \in \supp(v)$, by combining Theorem~\ref{thm:specialHilbertMumford} with~\eqref{eq:1psgLimit}. By Corollary~\ref{cor:Gordan}, this is equivalent to $0 \notin \Delta_A(v)$.
%	Hence $\lambda$ defines a hyperplane 
%	\[ H_{\lambda} = \{ x \in \RR^d \mid \langle \lambda, x \rangle = 0 \} \]
%	that separates zero from $\Delta_A(v)$. By Farkas' lemma, see \cite[Section~7.3]{SchrijverBook}, such a hyperplane exists if and only if $0 \notin \Delta_A(v)$.%todo
	
	For (c), we first prove that if $0$ is on the boundary of $\Delta_A(v)$, then $v$ is not polystable. We construct a point in the orbit closure of $v$, with support strictly smaller than that of $v$, and hence deduce that the orbit of $v$ is not closed. 
	Since $0$ lies on the boundary of $\Delta_A(v)$, it is contained in a minimal face $F \subsetneq \Delta_A(v)$. Since $A$ has integer entries, there is a hyperplane
		\[ H_{\lambda} := \{ x \in \RR^d \mid \langle \lambda, x \rangle = 0 \} , \]
	with $\lambda \in \ZZ^d$, such that $F = H_{\lambda} \cap \Delta_A(v)$. We choose the sign of $\lambda$ so that it has non-negative inner product with all of $\Delta_A(v)$. This ensures that the limit $w := \lim_{t \to 0} \lambda(t) \cdot v$ exists.
	The limit $w$ has $\supp(w) \subsetneq \supp(v)$, since $\Delta_A(w) \subseteq F$. Hence $w \in \overline{T \cdot v} \backslash T \cdot v$, and $T \cdot v$ is not closed.
	
	For the converse direction of (c), we show that if $v$ is semistable but not polystable, then $0 \notin {\rm relint} (\Delta_A(v))$. Let $w' \in \overline{T \cdot v} \backslash T \cdot v$. There exists $\lambda \in \ZZ^d$ such that $w := \lim_{t \to 0} \lambda(t) \cdot v \in T \cdot w'$, by Theorem~\ref{thm:generalHilbertMumfordTorus}. We have $\supp(w) \subseteq \supp(v)$ and, moreover, $\supp(w) \subsetneq \supp(v)$ (otherwise $w=v$ by~\eqref{eq:1psgLimit}, a contradiction). 
	Hence $\langle \lambda, A_j \rangle > 0$ for all $j \in \supp(v) \backslash \supp(w)$, while $\langle \lambda, A_j \rangle = 0$ for all $j \in \supp(w)$, by~\eqref{eq:1psgLimit}. We obtain $\Delta_A(v) \nsubseteq H_{\lambda}$ and $\Delta_A(w) = H_{\lambda} \cap \Delta_A(v)$, i.e., $\Delta_A(w)$ is a proper face of $\Delta_A(v)$. We have $T \cdot w = T \cdot w' \subseteq \overline{T \cdot v}$ and so $w$ is semistable as $v$ is semistable. By~(b), $0 \in \Delta_A(w)$ and hence $0$ is on the boundary of $\Delta_A(v)$.
	
	To prove (d), we can assume $v$ is polystable, i.e., $0 \in \mathrm{relint}(\Delta_A(v))$. We want to show that the dimension of the stabilizer $T_v = \{ t \in T \mid t \cdot v = v\}$ is zero if and only if the interior of $\Delta_A(v)$ equals its relative interior (i.e., if and only if $\Delta_A(v)$ is full-dimensional). Since $0 \in \Delta_A(v)$, the equality of the interior and relative interior holds if and only if $U:= \mathrm{span} \{ A_j \mid j \in \supp(v) \}$ equals $\RR^d$. If $T_v$ is positive dimensional, it must contain a one-parameter subgroup, i.e., some $\lambda \in \ZZ^d \backslash \{0\}$ with $\lambda(t) \cdot v = v $ for all $t \in \CC^\times$. Then $\langle \lambda, A_j \rangle = 0$ for all $j \in \supp(v)$, so the orthogonal complement $U^{\perp} \subseteq \RR^d$ contains a line, and $U \neq \RR^d$. Conversely, if $U \neq \RR^d$ then there exists non-zero $\lambda \in U^\perp$, which can be chosen to have integer entries, since $A$ has integer entries. Then the image of the non-trivial one parameter subgroup $\lambda$ lies in $T_v$, which is therefore positive-dimensional.
	
	Finally, if $T$ acts on $\CC^m$ by matrix $A \in \ZZ^{d \times m}$ with linearization $b \in \ZZ^d$, then this is the same as the action given by matrix $A' \in \ZZ^{d \times m}$, where $A'$ has $j^{th}$ column $A_j - b$; see \eqref{eq:torusAction} in Example~\ref{ex:GeneralGTaction}. Therefore, we can deduce the last statement by noting that $\Delta_{A'}(v) = \Delta_A(v) - b$.
\end{proof}

%remark on "general situation", i.e., "general" weight polytopes



%OLD
%for torus actions over \CC (general statement and proof as in Kraft's book); prove it!; stress relation to Farkas' Lemma
%do the weight polytope version 

%state most general version for reductive over \CC (and actually any perfect field due to Kempf; over \RR due to Birkes)
%example (e.g., Popov Vinberg) that shows that general torus statement does not generalize






%=========== Kempf-Ness ========================

\section{Kempf-Ness Theorem} \label{sec:KempfNess}


In this section we present an important analytical tool from invariant theory -- the Kempf-Ness Theorem. It plays a crucial role in this thesis and is heavily used both in Part~\ref{part:CompComplexity} and Part~\ref{part:AlgebraicStatistics}. The presentation is based on \cite{GradflowArXiv} and \cite{WeightMargin}, sometimes also on \cite{SiagaPaper} and \cite[Appendix~B]{DiscretePaper}.

First, we introduce the Setting~\ref{set:MomentMap} and define the moment map. Thereby, we follow the conventions used in \cite{GradflowArXiv} for $\KK=\CC$, which enables a good comparison with that paper in Part~\ref{part:CompComplexity}.
Afterwards, we compute the moment map in several examples. We continue with Kempf-Ness, Theorem~\ref{thm:KempfNessAKRS}, and deduce several statements from it. Finally, we introduce moment polytopes, which are induced by the moment map and generalize the concept of weight polytopes.

The literature on Kempf-Ness, moment maps and polytopes, and related topics is vast. The following list is certainly incomplete. We refer to \cite{KempfNess, MumfordGITbook, Wallach} for Kempf-Ness over $\CC$ and to \cite{RichardsonSlodowy, biliotti2021RealKempfNess, RealGIT, Wallach} for Kempf-Ness over $\RR$. Moment polytopes are treated in \cite{brion1987sur, GuilleminSternberg, kirwan1984convexity, osheaSjamaar2000moment, paradan2020moment} and related topics can be found e.g., in 
\cite{heinzner2007cartan, heinzner2008stratifications, KirwanBook, MumfordGITbook, marian2001on, NessStratification, thomas2006notes} and the references therein.

%\cite{KempfNess, NessStratification, MumfordGITbook, RichardsonSlodowy, RealGIT, biliotti2021RealKempfNess, marian2001on, heinzner2007cartan, heinzner2008stratifications, paradan2020moment, Wallach}


\subsubsection{The Moment Map}


We need the following fact, see \cite[Proposition~4.6]{KnappBook} or \cite[Theorem~2.9]{Wallach}.\footnote{The proof via the Haar measure also works for $\KK = \RR$.}

\begin{lemma} \label{lem:KinvariantInnerProduct}
	Let $K$ be a compact matrix Lie group and let $\pi \colon K \to \GL(V)$ be a continuous group morphism, where $V$ is a finite dimensional $\KK$-vector space. Then there exists an inner product $\langle \cdot, \cdot \rangle$ on $V$ such that $K$ acts isometrically, i.e.,
		\[ \forall \,  k \in K, \, v,w \in V \colon \quad  \langle \pi(k)v, \pi(k)w \rangle = \langle v, w \rangle . \]
	Equivalently, for all $k \in K$ we have $\pi(k)\adj = \pi(k)^{-1} (= \pi(k\HT))$, where $\pi(k)\adj$ denotes the adjoint of $\pi(k)$ with respect to $\langle \cdot, \cdot \rangle$.
\end{lemma}

We are now ready to fix the required data for defining a moment map.

\begin{setting}\label{set:MomentMap}
	Let $G \subseteq \GL_N(\KK)$ be a Zariski closed self-adjoint subgroup.
	Recall the following from Proposition~\ref{prop:SelfAdjointProperties}. The group $K := \{ g \in G \mid g\HT g = \Id_N \}$ is a maximal compact subgroup and setting $\pfrak := \Lie(G) \cap \Sym_N(\KK)$ we have an orthogonal decomposition $\Lie(G) = \Lie(K) \oplus \pfrak$ of \emph{real} vector spaces with respect to the inner product $\mathrm{Re} \big( \tr(X\HT Y) \big)$ on $\CC^{N \times N}$. Furthermore, $\pfrak = \imag \Lie(K)$ if $\KK = \CC$.
	%Moreover, if $\KK = \CC$ then $T := (\GT_m(\CC) \cap G)^\circ$ is a maximal torus of $G$ and $T_K := T \cap K$ is a maximal compact torus of $K$.
	
	Let $\pi \colon G \to \GL(V)$ be a rational representation defined over $\KK$ with differential $\Pi \colon \Lie(G) \to \End(V)$. Fix an inner product $\langle \cdot, \cdot \rangle$ on $V$ such that $K$ acts isometrically, and $\Pi(X)$ is self-adjoint
	for all $X \in \pfrak$.\footnote{In concrete representations this will usually be the standard inner product; except for polynomial scaling in Section~\ref{sec:PolynomialsGap}, where one has to take the Bombieri-Weyl inner product.}
	
	A $K$-invariant inner product always exists by Lemma~\ref{lem:KinvariantInnerProduct}. If $\KK = \CC$ then the property on $\Pi(X)$ automatically follows from the $K$-invariance of $\langle \cdot , \cdot \rangle$, $\CC$-linearity of $\Pi$ and the fact that $\pfrak = \imag \Lie(K)$. If $\KK = \RR$ the existence of $\langle \cdot, \cdot \rangle$ is ensured by \cite[Proposition~13.5]{BorelHarishChandra}, also compare \cite[§2.3]{RichardsonSlodowy}.
	\hfill\defnSymbol
\end{setting}

%\textbf{Comment for Peter:} It is crucial for Kempf Ness to have $\Pi(X)$ self-adjoint for all $X \in \pfrak$. Over $\CC$, this follows automatically from the $K$-invariance of $\langle \cdot , \cdot \rangle$, $\CC$-linearity of $\Pi$ and the fact that $\pfrak = \imag \Lie(K)$. Over $\RR$, I was not able to ensure this just from the $K$-invariance. Perhaps, it is not true in general as Richardson and Slodowy explicitly mention  \cite[§2.3]{RichardsonSlodowy}, which is \cite[Proposition~13.5]{BorelHarishChandra}. %todo
%An annoying thing about the real Kempf-Ness is that all other references \cite{biliotti2021RealKempfNess, RealGIT, Wallach} work in the much nicer situation $G \subseteq \GL(V)$, but do not mention when and how it is possible to transfer from a general rep to its image. There seem to be several issues in the real case that can happen... Long story short: the above setting is the only one where I can ensure that everything works.

We illustrate the general setting in an Example.

\begin{example}\label{ex:MomentMapSetting}
	Let $G := \SL_m(\KK)^d$ be block-diagonally embedded in $\GL_{dm}(\KK)$ ($N = dm$). Depending on $\KK \in \{\RR, \CC\}$, we have $K = \SO_m(\RR)^d$ or $K = (\SU_m)^d$, again block-diagonally embedded in $\GL_{dm}(\KK)$. Hence, their Lie algebras are block diagonally embedded into $\KK^{dm \times dm}$.
	Consider the tensor scaling action $\pi_{m,d}$ of $G$ on $V = (\CC^m)^{\otimes d}$ from Example~\ref{ex:RepTensorScaling}. For simplicity, let $d=3$. One verifies that for all $(X,Y,Z) \in \Lie(G)$
		\[ \Pi(X,Y,Z) = X \otimes \Id_m \otimes \Id_m + \Id_m \otimes Y \otimes \Id_m + \Id_m \otimes \Id_m \otimes Z \]
	using the Kronecker product of matrices. As desired, $\Pi(X,Y,Z) \in \Sym_{3m}(\KK)$ whenever $(X,Y,Z) \in \pfrak$.
	One verifies that $K$ acts isometrically on $V$ with respect to the standard inner product.
	\hfill\exSymbol
\end{example}


Given the above setting, remember from Definition~\ref{defn:StabilityGroupTopological} that a vector $v$ is called unstable if its capacity
	\[ \capac_G(v) := \inf_{g \in G} \; \| \pi(g) v \|^2 = \inf_{g \in G} \; \| g \cdot v \|^2  \]
equals zero. It is semistable if the capacity is positive. Considering for $v \in V \backslash \{0\}$ the so-called \emph{Kempf-Ness function}\index{Kempf-Ness function}
	\begin{equation} \label{eq:KempfNessFunction}
		F_v \colon G \to \RR, \quad v \mapsto \log \| \pi(g)v \| = \frac{1}{2} \log \big( \| \pi(g)v \|^2 \big)
	\end{equation}
we see that $v$ is semistable if and only if $F_v$ is bounded from below. In particular, if $v$ is semistable then intuitively the differential of $F_v$ should vanish. To make the concept of differential/gradient more precise, notice that $F_v$ is right-$G$-equivariant, i.e.,
	\[ \forall \, g,h \in G \colon \quad F_v(gh) = \log \| \pi(gh) v \|  = \log \| \pi(g) \pi(h) v\| = F_{\pi(h)v} (g) \, . \]
Furthermore, as $K$ acts isometrically on $V$ the function $F_v$ is left-$K$-invariant:
	\[ \forall\, k \in K, g \in G \colon \quad F_v(kg) = \log \| \pi(kg)v \| = \log \| \pi(g)v \| = F_v(g) \, . \]
The $G$-equivariance ensures that it is enough to consider the differential of $F_v$ at the identity. The latter is the map
	\[ \Lie(G) = \Lie(K) \oplus \pfrak \to \Lie(\RR) = \RR, \quad X \mapsto \left. \frac{d}{dt} \right|_{t=0} F_v \big( e^{tX} \big) \, , \]
compare Theorem~\ref{thm:Differential}. Now, the $K$-invariance of $F_v$ implies that the differential is zero on the direct summand $\Lie(K)$, so it suffices to consider the orthogonal complement $\pfrak$.
Altogether, we define the moment map \emph{both} in the real and complex case as the gradient of $F_v$.\footnote{This definition agrees with \cite[Definition~3.2]{GradflowArXiv} and \cite[Definition~4.1]{WeightMargin}, where only the complex case is considered.}

\begin{defn}[Moment Map] \label{defn:MomentMap}
	Consider the Setting~\ref{set:MomentMap} and define the \emph{moment map}\index{moment map} $\gls{muG} \colon V \backslash \{0\} \to \pfrak$ as follows.
	For $v \in V \backslash \{ 0 \}$, $\mu_G(v)$ is the unique element of the real vector space $\pfrak$, which satisfies for all $X \in \pfrak$
	\begin{align*}
		\tr ( \mu_G(v) X ) = \left.  \frac{d}{dt} \right\vert_{t=0} F_v \big( e^{tX} \big) = \frac{\langle v, \Pi(X)v \rangle}{\langle v,v \rangle}.
	\end{align*}
	Here we use that the inner product on $\pfrak$ is $\mathrm{Re} \big( \tr ( \mu_G(v)\HT X ) \big) = \tr ( \mu_G(v) X )$, that $\Pi(X)$ is $\RR$-linear and that $\langle \cdot, \cdot \rangle$ is linear in the second component.\footnote{Remember that, by our convention, inner products on $\CC$-vector spaces are always linear in the second component and semi-linear in the first.}
	\hfill\defnSymbol
\end{defn}

\begin{remark}
	In the literature $\mu_G(v)$ is often the differential of $F_v$ rather than the gradient. We follow the conventions in \cite{GradflowArXiv} for an easier comparison in Part~\ref{part:CompComplexity}.
\hfill\remSymbol
\end{remark}

%The maps $\mu_G$ and $\mu_{T}$ are indeed moment maps in the sense of symplectic geometry; namely for the induced action of $K$ and, respectively, $T_K$ on the projective space $\PP(V)$. Recall $\imag \Lie(K) \subseteq \CC^{dn \times dn}$ so we can consider $\| \mu_G(v) \|_F$ and $\| \mu_{T}(v) \|_F$. Note that $\mu$ is invariant under scalar multiples of $v$. %todo

Restricting $\pi$ to some Zariski closed self-adjoint subgroup $H \subseteq G$ we can similarly define the moment map $\mu_{H} \colon V \setminus \{0\} \to \qfrak$, where $H_K := H \cap K$ and $\qfrak := \Lie(H_K) \cap \Sym_N(\KK) \subseteq \pfrak$.
The moment maps are related as follows.

\begin{prop}[based on {\cite[Proposition~4.2]{WeightMargin}}]  \label{prop:MomentMaps}
	Let $p \colon \pfrak \to \qfrak$ be the orthogonal projection with respect to the inner product $\mathrm{Re}(\tr(X\HT Y)) = \tr(XY)$ on $\Sym_N(\KK)$. Then $\mu_{H} = p \circ \mu_G$ and $\norm{\mu_{H}(v)}_F \leq \norm{\mu_{G}(v)}_F$ for all $v \in V \setminus \lbrace 0 \rbrace$.
\end{prop}

\begin{proof}
	Since $\qfrak \subseteq \pfrak$ the definition of the moment maps gives
		\[ \tr(\mu_{H}(v) X) = \frac{\langle v, \Pi(X)v \rangle}{\langle v,v \rangle} = \tr(\mu_{G}(v) X) = \tr \big( p(\mu_G(v)) X \big) \]
	for all $X \in \qfrak$. Therefore, $p(\mu_G(v)) = \mu_{H}(v)$ and $\norm{\mu_{H}(v)}_F \leq \norm{\mu_{G}(v)}_F$ follows directly from this.
\end{proof}

Another property of the moment map is its $K$-equivariance.

\begin{prop}\label{prop:UnitaryEquivarianceMomentMap}
	For all $v \in V \backslash \{0\}$ and all $k \in K$, $\mu_G(k\cdot v) = k \mu_G(v) k\HT$.
\end{prop}

\begin{proof}
	Fix $v \in V \backslash \{0\}$ and $k \in K$. Note that $k \pfrak k\HT = \pfrak$.
	For all $X \in \pfrak$,
		\begin{align*}
			\tr \big( \mu_G(k \cdot v) X \big) &= \frac{1}{\| v \|^2} \big\langle \pi(k) v, \Pi(X) \pi(k) v \big\rangle
			\overset{(*)}{=} \frac{1}{\| v \|^2} \big\langle v, \Pi(k\HT X k ) v \big\rangle \\
			&= \tr \big( \mu_G(v) k\HT X k \big) = \tr \big( k \mu_G(v) k\HT X \big) ,
		\end{align*}
	where we used in $(*)$ that $\pi(k)\adj = \pi(k\HT)$ and then Theorem~\ref{thm:Differential} Item~1. Since $k \mu_G(v) k\HT \in \pfrak$ we must have $\mu_G(k \cdot v) = k \mu_G(v) k\HT$.
\end{proof}





%Examples of Moment Maps
\subsubsection{Moment Map in Examples}

We state the moment maps for several actions and give a detailed computation in some cases. At a first read one may only skim through the results to quickly progress to the Kempf-Ness Theorem.

\begin{example}[Torus Actions] \label{ex:MomentMapTorus}
	Consider a complex torus $T \subseteq \GT_N(\CC)$ and set $T_K := \{ t \in T \mid t\HT t = \Id_N\}$. Let $\pi \colon T \to \GL(V)$ be a rational representation. Then $\pi$ admits a weight space decomposition $V = \bigoplus_{\omega \in \Omega(\pi)} V_\omega$, where $\Omega(\pi) \subseteq \imag \Lie(T_K)$ is the set of weights; compare Theorem~\ref{thm:WeightSpaceDecomposition}. Equip $V$ with an inner product as in Setting~\ref{set:MomentMap}. We show that the weight spaces are pairwise orthogonal. Let $\omega, \eps \in \Omega(\pi)$ and choose $v_\omega \in V_\omega$, $v_\eps \in V_{\eps}$. As $\pfrak = \Lie(T) \cap \Sym_N(\CC) = \imag \Lie(T_K)$ acts via self-adjoint operators and $v_\omega$, $v_\eps$ are weight vectors (Definition~\ref{defn:Weights}), we compute for all $X \in \pfrak$
		\[ \tr(X\omega) \langle v_\omega, v_\eps \rangle = \langle \Pi(X)v_\omega, v_\eps \rangle 
		= \langle v_\omega, \Pi(X) v_\eps \rangle = \tr(X\eps) \langle v_\omega, v_\eps \rangle .\]
	If $\langle v_\omega, v_\eps \rangle \neq 0$, then $\tr(X\omega) = \tr(X\eps)$ holds for all $X \in \pfrak$. Since $\omega, \eps \in \pfrak$ we necessarily have $\omega = \eps$ by non-degeneracy of the trace inner product on $\pfrak$. By contraposition, distinct weight spaces are orthogonal. Therefore, writing $v = \sum_\omega v_\omega \in V$ we have for all $X \in \pfrak$ that
		\begin{align*}
			\tr \big( \mu_T(v) X \big) &= \frac{1}{\|v\|^2} \langle v, \pi(X) \sum_{\omega} v \rangle
			= \frac{1}{\|v\|^2} \Big\langle \sum_\eps v_\eps, \sum_{\omega} \tr(\omega X) v_\omega \Big\rangle \\
			&= \frac{1}{\|v\|^2} \sum_{\omega} \tr(\omega X) \langle v_\omega, v_\omega \rangle
			= \tr \left( \sum_{\omega} \frac{\|v_\omega\|^2}{\|v\|^2} \omega X \right) .
		\end{align*}
	Hence, the moment map at $v$ is given by
		\begin{equation}\label{eq:MomentMapGeneralTorus}
			\mu_T(v) = \sum_{\omega \in \Omega(\pi)}  \frac{\|v_\omega\|^2}{\|v\|^2} \, \omega \, .
		\end{equation}
	
	Let us end by specifying this in two special cases. First, let $T = \GT_d(\CC)$ act on $V = \CC^m$ via the matrix $A \in \ZZ^{d \times m}$ with linearization $b \in \ZZ^d$ as in Example~\ref{ex:GeneralGTaction}. Then $e_j \in \CC^m$ is a weight vector for the weight $A_j - b$, where $A_j$ denotes the $j^{th}$ column of $A$.  For $v = (v_1,\ldots,v_m) \in \CC^m$, define the vector $v^{[2]} := (|v_1|^2, \ldots, |v_m|^2)$. %todo refer here for v^{[2]}
	Then \eqref{eq:MomentMapGeneralTorus} becomes
		\begin{equation}\label{eq:MomentMapTorusA-b}
			\mu_T(v) = \sum_{j=1}^m \frac{|v_j|^2}{\|v\|^2} \, (A_j - b) = \frac{1}{\|v\|^2} \, Av^{[2]} - b
			= \frac{1}{\|v\|^2} \big( Av^{[2]} - \|v\|^2 b \big) \, .
		\end{equation}
	
	Second, consider the matrix scaling action from Example~\ref{ex:RepTensorScaling}: $T = \ST_m(\CC)^2$ acts on $\CC^m \times \CC^m \cong \CC^{m \times m}$ via $\pi_{m,2}$. We know from Example~\ref{ex:WeightsTensorScaling} that $(\eps_i,\eps_j) \in (\onePerp)^2$ is a weight with weight vector $e_i \otimes e_j \cong E_{ij}$. Therefore, for $v = (v_ij) \in \CC^{m \times m}$ \eqref{eq:MomentMapGeneralTorus} becomes
		\[ \mu_T(v) = \sum_{i,j=1}^m \frac{|v_{ij}|^2}{\|v\|^2} (\eps_i, \eps_j) = \frac{1}{\|v\|^2} \sum_{i,j=1}^m |v_{ij}|^2 \big( (\eps_i,0) + (0, \eps_j) \big)  \]
	Setting $M_v := (|v_{ij}|^2)_{i,j} \in \CC^{m \times m}$, we compute that
		\begin{align*}
			\sum_{i,j=1}^m |v_{ij}|^2 (\eps_i,0) = \sum_{i=1}^m (M_v)_{i,+} (e_i - m^{-1} \ones_m, 0)
			= \left( \sum_{i=1}^m (M_v)_{i,+}e_i - \frac{M_{+,+}}{m} \ones_m, 0 \right) .
		\end{align*}
	Note that $M_{+,+} = \|v\|^2$ and that $\sum_i (M_v)_{i,+}e_i$ is the vector of row sums of $M_v$, which we denote by $r(M_v)$. A similar computation to the above holds for $c(M_v)$, the vector of column sums $(M_v)_{+,j}$. Altogether, we deduce that
	\begin{equation}\label{eq:MatrixScalingMomentmap}
		\mu_T(v) = \frac{1}{\|v\|^2} \left( r(M_v) - \frac{\|v\|^2}{m} \ones_m, \, c(M_v) - \frac{\|v\|^2}{m} \ones_m \right).
	\end{equation}
	is  the moment map at $v$ for matrix scaling.
	\hfill\exSymbol
\end{example}

\begin{example}[Left Multiplication] \label{ex:MomentMapLeftMult}
	Let $\pi$ be the action of $G = \SL_m(\KK)$ on $V = \KK^{m \times n}$ via left-multiplication. Then $K$ acts isometrically on $V$ with respect to the Frobenius inner product. Moreover, for $X \in \Lie(G)$ and $Y \in V$ we have $\Pi(X)Y = XY$. In particular, $\Pi(X)$ is self-adjoint for $X \in \pfrak$. We compute for all $X \in \pfrak$
		\begin{align*}
			\tr ( \mu_G(Y) X ) &= \frac{1}{\|Y\|^2} \langle Y, \Pi(X)Y \rangle = \frac{1}{\|Y\|^2} \tr( Y\HT X Y) \\
			&\overset{(*)}{=} \frac{1}{\|Y\|^2} \tr( Y Y\HT X) - \frac{1}{m} \tr(X)	= \tr \left( \bigg(\frac{Y Y\HT}{\|Y\|^2} - \frac{1}{m} \Id_m \bigg) X \right)
		\end{align*}
	where we used $\tr(X) = 0$ in $(*)$. Note that $\tr(YY\HT / \|Y\|^2) = 1$ and hence it cannot be $\mu_G(Y) \in \pfrak$. However, subtracting $m^{-1}\Id_m$ ensures we get a trace zero matrix in $\Sym_m(\KK)$, i.e., a matrix in $\pfrak = \Lie(G) \cap \Sym_m(\KK)$. Therefore,
		\begin{equation}\label{eq:LeftMultMomentMap}
			\mu_G(Y) = \frac{Y Y\HT}{\|Y\|^2} - \frac{1}{m} \Id_m = \frac{1}{\| Y\|^2} \left( Y Y\HT - \frac{\|Y\|^2}{m} \Id_m \right)
		\end{equation}
	gives the moment map.
	\hfill\exSymbol
\end{example}


\begin{example}[Action on a Quiver] \label{ex:MomentMapQuiver}
	Consider the quiver $Q$
		\begin{equation}\label{eq:ThreeSinkQuiver}
			\begin{tikzcd}
				1 \ar[r, "B_1"] & 2 & 3 \ar[l, "B_2" ']
			\end{tikzcd}
		\end{equation}
	with dimension vector $\alpha = (m,m,m)$. The labels in \eqref{eq:ThreeSinkQuiver} indicate how $(B_1,B_2) \in V = \Rscr(Q,\alpha) = (\KK^{m \times m})^2$ is associated to the arrows. A group element $g \in G = \SL_\alpha(\KK) = \SL_m(\KK)^3$ acts on $V$
	via
		\[ (g_1,g_2,g_3) \cdot (B_1, B_2) = (g_2 B_1 g_1^{-1}, g_2 B_2 g_3^{-1}) , \]
	compare Example~\ref{ex:QuiverRep}. Let $\pi$ be the corresponding representation. 
	Equipping $V$ with the standard inner product\footnote{That is, the two copies $\KK^{m \times m}$ are orthogonal to each other and each copy is equipped with the Frobenius inner product.} the group $K$ acts isometrically on $V$. Recall that we think of $G$, $K$ and their Lie algebras as block diagonally embedded into $\GL_{3m}(\KK)$ respectively $\KK^{3m \times 3m}$.\footnote{For convenience, this is neglected in the notation; e.g., we write $X = (X_1,X_2,X_3) \in \Lie(G) \cong \Lie(\SL_m(\KK))^3$ instead of $X = \diag(X_1,X_2,X_3)$.}
	For $A \in \KK^{m \times m}$ set
		\begin{equation}\label{eq:PhiMomentMapTau}
			\Phi_1(A) := - A\HT A + \frac{\|A\|^2_F}{m} \Id_m  \qquad \text{and} \qquad
			\Phi_2(A) := A A\HT - \frac{\|A\|^2_F}{m} \Id_m .
		\end{equation}
	which are in $\Sym_m(\KK)$ and have trace zero as $\tr(A\HT A) = \tr(A A\HT) = \|A\|^2_F$. Therefore, $\Phi_1(A), \Phi_2(A) \in \qfrak := \Lie(\SL_m(\KK)) \cap \Sym_m(\KK)$.
	We will show that the moment map is given by
		\begin{equation}\label{eq:SinkMomentMap}
		\mu_G(B) = \frac{1}{\|B\|^2} \, \big( \Phi_1(B_1), \Phi_2(B_1) + \Phi_2(B_2), \Phi_1(B_2) \big) \, .
		\end{equation}
	
	First, note that for general $A \in \KK^{m \times m}$ and $(X_1,X_2) \in \Lie(\SL_m(\KK))^2$ we have
	 \[ \left. \frac{d}{dt} \right|_{t=0} e^{tX_1} A e^{-tX_2}
	 = \left. \left( X_1 e^{t X_1} A e^{-t X_2} + e^{t X_1} A (-X_2) e^{-t X_2} \right)\right|_{t=0} 
	 = X_1 A - A X_2 \, .\]
	 Therefore, $X = (X_1,X_2,X_3) \in \Lie(G)$ acts via
	 	\[ \Pi(X_1,X_2,X_3)(B_1,B_2) = ( X_2 B_1 - B_1 X_1, X_2 B_2 - B_2 X_3 ) . \]
	 In particular, $\pfrak$ acts via self-adjoint operators on $V$. 
	 By Definition~\ref{defn:MomentMap}, the moment map $\mu_G(B) = \|B\|^{-2} \big( \mu_1(B), \mu_2(B), \mu_3(B) \big) \in \pfrak \cong \qfrak^3$, is determined by 
	 \begin{equation}\label{eq:SinkMomentMapCondition}
	 	\begin{split}			
	 		\sum_{i=1}^3 \tr \big( \mu_i(B) X_i \big) &= \big\langle B, \Pi(X) B \big\rangle
	 		%= \big\langle (B_1, B_2), ( X_2 B_1 - B_1 X_1, X_2 B_2 - B_2 X_3 ) \big\rangle
	 		\\	&= \tr \big( B_1\HT (X_2 B_1 - B_1 X_1)  \big) + \tr\big( B_2\HT (X_2 B_2 - B_2 X_3 ) \big)
	 	\end{split}
	 \end{equation}
	 for all $X = (X_1,X_2,X_3) \in \pfrak$.
	 Thus, using $X = (X_1,0,0) \in \pfrak$ we obtain with $\tr(X_1)=0$ that
	 \begin{align*}
	 	\tr \big( \mu_1(B) X_1 \big)& = \tr \big( B\HT_1 (-B_1 X_1) \big) 
	 	\overset{(*)}{=}  \tr \big( - B\HT_1 B_1 X_1  \big) + \frac{\|B_1\|_F^2}{m} \tr( X_1 ) \\
	 	&= \tr \big( \Phi_1(B_1) X_1 \big) .
	 \end{align*}
	 Since $\Phi_1(B_1) , \mu_1(B) \in \qfrak$, we deduce $\mu_1(B) = \Phi_1(B_1)$ by non-degeneracy of the trace inner product on $\qfrak$. Similarly, one shows $\mu_3(B) = \Phi_1(B_2)$. 
	 Finally, for $X = (0, X_2, 0) \in \pfrak$ in Equation~\eqref{eq:SinkMomentMapCondition} we obtain
	 	\begin{align*}
	 		\tr \big( \mu_1(B) X_2 \big) &= \tr \big( B_1\HT X_2 B_1 \big) + \tr \big( B_2\HT X_2 B_2 \big) \\
	 		&=  \tr \big( B_1 B_1\HT X_2 \big) + \tr \big( B_2 B_2\HT X_2 \big) - \frac{\|B_1\|^2_F + \|B_2\|^2_F}{m} \tr(X_2) \\
	 		&= \tr \big( (\Phi_2(B_1) + \Phi_2(B_2)) X_2 \big) \, .
	 	\end{align*}
	 We deduce $\mu_2(B) = \Phi_2(B_1) + \Phi_2(B_2)$ and hence \eqref{eq:SinkMomentMap} holds.
	 
	Analogously, one can consider $\alpha = (m,m,m)$, the quiver $Q'$
	\begin{equation}\label{eq:ThreeSourceQuiver}
	 	\begin{tikzcd}
	 		1 & 2 \ar[l, "C_1" '] \ar[r, "C_2"] & 3 
	 	\end{tikzcd}
	\end{equation}
	and its associated action of $G = \SL_m(\KK)^3$ on $V = \Rscr(Q', \alpha) = (\KK^{m \times m})^2$. In that case, $g \in G$ acts on $C = (C_1,C_2) \in V$ via $g \cdot C = (g_1 C_1 g_2^{-1} , g_3 C_2 g_2^{-1})$ and
		\begin{equation}\label{eq:SourceMomentMap}
	 		\mu_G(C) = \frac{1}{\|C\|^2} \, \big( \Phi_2(C_1), \Phi_1(C_1) + \Phi_1(C_2), \Phi_2(C_2) \big) \, .
		\end{equation}
	is the moment map at $C$.
	\hfill\exSymbol
\end{example}

\begin{example}[Left-Right Action] \label{ex:MomentMapLeftRight}
	Consider the left-right action of $G = \SL_{m_1}(\KK) \times \SL_{m_2}(\KK)$ on $V = (\KK^{m_1 \times m_2})^n$ from Example~\ref{ex:RepLeftRight}. One computes that
	\begin{equation}\label{eq:MomentMapLeftRight}
		\mu_G(Y) = \frac{1}{\|Y\|^2} \left( \sum_{i=1}^n Y_i Y_i\HT - \frac{\|Y\|^2}{m_1} \Id_{m_1}, \bigg( \sum_{i=1}^n Y_i\HT Y_i \bigg)\T -  \frac{\|Y\|^2}{m_2} \Id_{m_2} \right)
	\end{equation}
	is the moment map at $Y = (Y_1,\ldots,Y_n) \in V$.
	\hfill\exSymbol
\end{example}

\begin{example}[Tensor Scaling] \label{ex:MomentMapTensorScaling}
	Let $\pi_{m,d}$ be the natural action of $G = \SL_m(\KK)^d$ on $V = (\KK^m)^{\otimes d}$. For a tensor $v = (v_{i_1,\ldots,i_d}) \in V$, consider its flattenings $M_1,\ldots,M_d \in \KK^{m \times m^{d-1}}$ into the $d$ many directions, e.g., $(M_1)_{i_1,(i_2,\ldots,i_d)} = v_{i_1,\ldots,i_d}$. One can compute that the moment map of $\pi_{m,d}$ is given by
		\begin{equation}\label{eq:MomentMapTensor}
			\mu_G(v) = \frac{1}{\|v\|^2} \left( M_1 M_1\HT - \frac{\|v\|^2}{m} \Id_m, \ldots, M_d M_d\HT - \frac{\|v\|^2}{m} \Id_m \right) .
		\end{equation}
	The matrices $M_{l} M_l\HT$, $l \in [d]$ are called \emph{(one-body) quantum marginals}\index{quantum marginal} of~$v$.
	Usually, they are considered for $\KK = \CC$ and they play an important role in quantum information theory, see corresponding references in Section~\ref{sec:CompProblems}. Thus, Equation~\eqref{eq:MomentMapTensor} links invariant theory via tensor scaling to this research area.
	\hfill\exSymbol
\end{example}



\subsubsection{The Theorem of Kempf-Ness}

In the following we state the Kempf-Ness Theorem, which gives criteria to detect semi- and polystability. It was first proven by Kempf and Ness in \cite{KempfNess} over $\CC$. The real case is due to Richardson and Slodowy \cite{RichardsonSlodowy}, and their result allows to deduce the complex case as well \cite[Remark~4.5(d)]{RichardsonSlodowy}.

\medskip

%state that geodesic convexity of KempfNess function is main ingredient of proof %polar decomp and Kempf Ness also important

First, let us give some intuition for the statement. 
Remember that a vector $v$ is semistable if and only if the Kempf-Ness function $F_v$, see \eqref{eq:KempfNessFunction}, is bounded from below.
An important property of $F_v$ is its geodesic convexity on the manifold $P = \{ g\HT g \mid g \in G \}$ of positive definite matrices in $G$ \cite[Proposition~3.13]{GradflowArXiv}; also compare Theorem~\ref{thm:GmodKtotallyGeodesicSymmetric} and Example~\ref{ex:GeodesicConvexFunctions}. Similarly to convexity in the Euclidean sense, geodesic convex functions on $P$ achieve a global minimum at a point if and only if their gradient vanishes at the point.\footnote{For this, the facts from Theorem~\ref{thm:GmodKtotallyGeodesicSymmetric} that $P$ is a totally geodesic manifold and has non-positive curvature are crucial.}

There are several statements to which one refers as (part of) Kempf-Ness Theorem. We collect them in Theorem~\ref{thm:KempfNessAKRS}, whose formulation is based on \cite[Theorem~2.2]{SiagaPaper}.

\begin{theorem}[Kempf-Ness Theorem] \label{thm:KempfNessAKRS}
	\index{Kempf-Ness Theorem}
	Consider the Setting~\ref{set:MomentMap}. In particular, $G \subseteq \GL_N(\KK)$ is Zariski closed and self-adjoint and $K = \{g \in G \mid g\HT g = \Id_N \}$. Moreover, $\pi \colon G \to \GL(V)$ is a rational representation over $\KK$ with moment map~$\mu$.
	For $v \in V \backslash \{0\}$, we have:
	\begin{itemize} \itemsep 3pt
		\item[(a)] The vector $v$ is of minimal norm in its orbit if and only if $\mu(v)=0$.
		
		\item[(b)] Let $v$ be of minimal norm in its orbit. If $X \in \pfrak$ satisfies $\| e^X \cdot v \| = \|v\|$, then $X \cdot v = 0$. If $w \in G \cdot v$ is such that $\|v\| = \| w \|$, then $w \in K \cdot v$.
		
		\item[(c)] If the orbit $G \cdot v$ is closed, then there exists some $w \in G \cdot v$ with $\mu(w)=0$.
		
		\item[(d)] If $\mu(v)=0$, then the orbit $G \cdot v$ is closed.
		
		\item[(e)] The vector $v$ is polystable if and only if there exists $0 \neq w \in G \cdot v$ with  $\mu(w)=0$.
		
		\item[(f)] The vector $v$ is semistable if and only if there exists $0 \neq w \in \overline{G \cdot v}$ with $\mu(w)=0$.
	\end{itemize}
	We can replace $G$ by any Euclidean closed subgroup $H \subseteq G$ with $G^\circ \subseteq H$. In this case, $K$ is replaced by $K' = \{ h \in H \mid h\HT h = \Id_N\}$.
\end{theorem}

\begin{proof}
	For $\KK = \RR$: note that our Setting~\ref{set:MomentMap} fits into the framework of \cite{RichardsonSlodowy}. In the latter work, $G \subseteq \GL(E)$ is stable under a Cartan involution, which just means there is an inner product on $E$ to which $G$ is self-adjoint. For us, $E = \RR^N$ is equipped with the standard inner product. Our $\pfrak = \Lie(G) \cap \Sym_N(\RR)$ is the $-1$ eigenspace of $\theta \colon \Lie(G) \to \Lie(G), X \mapsto -X\T$, and hence agrees with the $\pfrak$ in \cite{RichardsonSlodowy}. Moreover, the inner product from Setting~\ref{set:MomentMap} is $K$-invariant and $\pi(X)$ is self-adjoint for all $X \in \pfrak$ as required by \cite[§3]{RichardsonSlodowy}.
	
	Now, Part~(a) is the equivalence of (i) and (iii) in \cite[Theorem~4.3]{RichardsonSlodowy}. Item~(b) is the last part of \cite[Theorem~4.3]{RichardsonSlodowy} plus Lemma~4.2, which ensures the statement on $X \in \pfrak$. \cite[Theorem~4.4]{RichardsonSlodowy} yields parts~(c), (d) and~(e).
	Finally, part~(f) follows from the fact that any orbit closure $\overline{G \cdot v}$ contains a unique closed orbit (\cite[Theoreme~2.7]{luna1975sur})\footnote{also see \cite[§9.3]{RichardsonSlodowy} or \cite[Theorem~1.1(iii)]{RealGIT}}, which is not the zero orbit if and only if $v$ is semistable.
	
	For $\KK = \CC$: by \cite[Remark~4.5(d)]{RichardsonSlodowy} it follows from the real case. Still, let us refer to the original paper \cite{KempfNess}.
	Parts~(a) respectively (b) are \cite[Theorem~0.1(a) respectively (b)]{KempfNess}, while \cite[Theorem~0.2]{KempfNess} yields items~(c), (d) and~(e).\footnote{Note that ``stable'' in \cite{KempfNess} means polystable in our sense.} Part~(f) again follows from the fact that any orbit closure $\overline{G \cdot v}$ contains a unique closed orbit, Theorem~\ref{thm:GeneratingInvariantsSeparate}. We note that the assumption in \cite{KempfNess} of $G$ being connected is unnecessary.\footnote{Indeed, \cite{RichardsonSlodowy} does not assume this.}
	
	For $H$ being a Euclidean closed subgroup with $G^\circ \subseteq H$, note that $H$ is self-adjoint by Corollary~\ref{cor:PolarDecompositionSubgroup}. Thus, for $\KK = \RR$ it follows from the general setting of \cite{RichardsonSlodowy}.\footnote{\cite{RichardsonSlodowy} also assumes $H$ to be Zariski dense, but this is only needed in \cite[§6]{RichardsonSlodowy} and not in §3 and §4 which prove Kempf-Ness. Alternatively, one can deduce the statement on $H$ from \cite{RealGIT}, see Remark~\ref{rem:KempfNessFurtherLiterature} below.}
	If $\KK = \CC$ note that $H$ is Zariski closed, because it consists of several connected components of $G$ that are all Zariski closed as $G^\circ = G^{\circ, \Zar}$ over $\CC$ (compare Section~\ref{sec:LinearAlgebraicGroups}). Hence, $H$ is Zariski closed and self-adjoint which puts us again in Setting~\ref{set:MomentMap}.
\end{proof}

\begin{remark}[Further Literature] \label{rem:KempfNessFurtherLiterature}
	Parts (a)--(d) of Theorem~\ref{thm:KempfNessAKRS} are the formulations of \cite[Theorems~3.26 and 3.28]{Wallach}. However, one needs to be careful: Wallach directly works with a Zariski closed self-adjoint subgroup of $\GL(V)$, but $\pi(G) \subseteq \GL(V)$ may not be Zariski closed for $\KK = \RR$, compare Example~\ref{ex:BorelRealPoints}.
	
	Still, if $\KK = \RR$ we know from Proposition~\ref{prop:LieGroupImage} that $\pi(G) \subseteq \GL(V)$ is a Euclidean closed Lie subgroup. Furthermore, in Setting~\ref{set:MomentMap} the inner product $\langle \cdot, \cdot \rangle$ on $V$ is $K$-invariant and for all $X \in \pfrak$ the operator $\Pi(X)$ is self-adjoint. Thus, the polar decomposition on $G$ induces a polar decomposition on $\pi(G)$ and hence $\pi(G)$ is self-adjoint with respect to $\langle \cdot, \cdot \rangle$. Altogether, $\pi(G) \subseteq \GL(V)$ satisfies the assumptions of \cite{biliotti2021RealKempfNess, RealGIT} and hence one can deduce Kempf-Ness over $\RR$ also from the formulations in \cite[Theorem~1.1]{RealGIT} respectively \cite[Theorem~1]{biliotti2021RealKempfNess}.
	\hfill\remSymbol
\end{remark}

For Computational Invariant Theory an important consequence of Kempf-Ness Theorem~\ref{thm:KempfNessAKRS}(f) is a ``duality'' between capacity and moment map:
\begin{equation}\label{eq:KempfNessDuality}
	\capac_G(v) = 0 \qquad \Leftrightarrow \qquad 
	0 < \inf_{g \in G} \; \| \mu_G(g \cdot v) \|_F = \min_{0 \neq w \in \overline{G \cdot v}} \; \| \mu_G(w) \|_F .
\end{equation}
We revisit this in Part~\ref{part:CompComplexity}, where we state a quantitive version in Theorem~\ref{thm:NonCommutativeDuality}.
Next, let us illustrate Kempf-Ness in an example.
%todo this generalizes linear programming duality??
%moment map for torus actions; relations to linear/geometric programming duality (together with Hilbert-Mumford)


\begin{example}\label{ex:MinimumForLeftMult}
	Consider the left multiplication of $G = \SL_m(\KK)$ on $V = \KK^{m \times n}$. We know from Example~\ref{ex:SLactionOnKmTimesn} that $Y \in V$ is either unstable or stable. The latter case happens if and only if $Y$ has full row rank. Now, assume that $Y$ is stable. To illustrate Kempf-Ness, Theorem~\ref{thm:KempfNessAKRS}, we determine an element of minimal norm in $G \cdot Y$ and, as a sanity check, show that the moment map vanishes.
	
	This problem is classical and we follow the explanations below Equation~(2.2) in \cite{burgisser2017alternating}.
	First, note that the AM-GM inequality for the eigenvalues of a positive semi-definite matrix $\Psi \in \KK^{m \times m}$ translates to
	$
	\tr(\Psi) \geq m ( \det(\Psi) )^{1/m}.
	$
	With this inequality we compute that for all $g \in \SL_m(\KK)$
	\[ \| g \cdot Y \|^2 = \tr \big( gY Y\HT g\HT \big) \geq m \big( \det \big( g Y Y\HT g\HT \big) \big)^{1/m}
	= m \det \big( Y Y\HT \big)^{1/m} .   \]
	Setting $M := Y Y\HT$, we have that $\capac_{G} (Y) \geq m \det(M)^{1/m}$. In fact, equality holds as follows. As $Y$ has full row rank the matrix $M$ is invertible, so $M \in \PD_m(\KK)$. Let $M^{1/2} \in \PD_m(\KK)$ be the square root and set $h := \det(M)^{1/(2m)} M^{-1/2} \in G$. We compute
		\begin{equation}\label{eq:MinimumForLeftMult}
			\begin{split}
				(hY)(hY)\HT &= \det(M)^{1/m} M^{-1/2} Y Y\HT M^{-1/2} \\
				&= \det(M)^{1/m} M^{-1/2} M M^{-1/2} = \det(M)^{1/m} \Id_m .
			\end{split}
		\end{equation}
	Therefore, $\| h\cdot Y \|^2 = \det(M)^{1/m} \tr(\Id_m) = m \det(M)^{1/m}$ and we necessarily have
		\[ \capac_G(Y) =  \| h \cdot Y \|^2 = m \det(M)^{1/m}. \]
	We see that $h \cdot Y$ is of minimal norm in $G \cdot Y$ and hence $Y$ is indeed polystable by Kempf-Ness Theorem~\ref{thm:KempfNessAKRS}. Using \eqref{eq:MinimumForLeftMult} and the value for $\| h\cdot Y \|^2$ we obtain
		\[ (hY)(hY)\HT - \frac{\|hY\|^2}{m} \Id_m = \det(M)^{1/m} \Id_m \, - \, \frac{m \det(M)^{1/m}}{m} \Id_m = 0. \]
	Hence, $\mu_G(h \cdot Y) = 0$ by Equation~\eqref{eq:LeftMultMomentMap} in Example~\ref{ex:MomentMapLeftMult}.
	\hfill\exSymbol
\end{example}

In the following we present three statements which fall into the realm of Kempf-Ness.

\begin{lemma}\label{lem:StabilizerSelfAdjoint}
	Consider the Setting~\ref{set:MomentMap}. Let $v \in V$ be of minimal norm in its orbit. Then the stabilizer $G_v$ is Zariski closed and self-adjoint.
\end{lemma}

\begin{proof}
	The same proof as for \cite[Corollary~2.25]{Wallach} applies. First, recall from Section~\ref{sec:LinearAlgebraicGroups} that $G_v$ is Zariski closed as the action via $\pi$ is algebraic. To show self-adjointness, use the polar decomposition (Theorem~\ref{thm:PolarDecomposition}) to write $g = k \exp(X) \in X \in G_V$ with $k \in K$ and $X \in \pfrak$. Then $X = X\HT$ yields $g\HT = \exp(X\HT) k^{-1} = \exp(X) k^{-1}$. Now, $g \in G_v$ and $K$ acting isometrically imply $\|v\| = \|g \cdot v\| = \| \exp(X) v\|$. Kempf-Ness Theorem~\ref{thm:KempfNessAKRS}(b) yields $\Pi(X)v = 0$ and hence $\pi(\exp(X))v = \exp(\Pi(X)) v = v$. That is, $\exp(X) \in G_v$ and thus $k = g \exp(X)^{-1} \in G_v$. Altogether, $g\HT \in G_v$.
\end{proof}


\begin{prop}\label{prop:GvsIdentityComponent}
	Let $G \subseteq \GL_m(\KK)$ be Zariski closed and self-adjoint with Euclidean identity component $G^\circ$. Set $K := \{ g \in G \mid g\HT g = \Id_m\}$.
	\begin{itemize}
		\item[(i)] Then there exist finitely many $k_1 = \Id_m, k_2, \ldots, k_l \in K$ such that the $k_i G^\circ$ are the Euclidean connected components of $G$.
		
		\item[(ii)] If $\pi \colon G \to \GL(V)$ is a rational representation over $\KK$ and $K$ acts isometrically on $V$ with respect to some inner product, then the stability notions for $G$ and $G^\circ$ coincide.
	\end{itemize}
\end{prop}

\begin{proof}
	%\textbf{Comment for Peter:} Actually, I have almost a full proof of part~(ii) for \emph{any} algebraic action of any algebraic group (just ignore the parts on $K$ in the statement). However, I am lacking ``$G$-polystable $\Rightarrow$ $G^\circ$-polystable''. ($G \cdot v$ is a finite union of disjoint $G^\circ$-orbits, which are all homeomorphic to each other. I do not know how to prove that these orbits have to be closed. Perhaps it is wrong, but a counter-example has to be quite complicated... )%todo delete later
	
	For part~(i), remember that $G$ has only finitely many Euclidean connected components, since it is algebraic. Moreover, $\exp(X) \in G^\circ$ for all $X\in \Lie(G)$, compare Proposition~\ref{prop:LieAlgebraProperties}. Therefore, the polar decomposition (Theorem~\ref{thm:PolarDecomposition}) yields part~(i).
	
	For part~(ii), let $v\in V$. First, we have $\capac_G (v) = \capac_{G^\circ}(v)$ using part~(i) and that $K$ acts isometrically on $V$. Thus, $v$ is $G$-unstable/semistable if and only is $v$ if $G^\circ$-unstable/semistable.
	
	For part(iii), note that we can apply Kempf-Ness to $G$ and $G^\circ$. Combining Theorem~\ref{thm:KempfNessAKRS}(a) and~(e) yields that $v$ is polystable if and only if its capacity is positive and attained. Since $K$ acts isometrically, part~(i) shows that $\capac_G (v) = \capac_{G^\circ}(v)$ is attained by some $g \in G$ if and only if it is attained by some $g' \in G^\circ$. Hence, $v$ is $G$-polystable if and only if it is $G^\circ$-polystable.
	%by applying Theorem~\ref{thm:KempfNessAKRS}(e) to the actions of $G$ and of $G^\circ$.
	%assume that $G^\circ \cdot v$ is closed, then $k_i G^\circ \cdot v$ is closed for all $i \in [l]$. Hence, $G \cdot Y$ is closed as the union of the finitely many closed sets $k_i G^\circ \cdot v$. Conversely, assume $G \cdot Y$ is closed. 
	
	Finally, to ensure the same for ``stable'' it suffices to show that $G_v$ is finite if and only if $(G^\circ)_v$ is finite. If $G_Y$ is finite, then $(G^\circ)_v$ is finite as $(G^\circ)_v \subseteq G_v$. For the converse, note that $(G_v)^\circ \subseteq G^\circ$ and hence $(G_v)^\circ \subseteq (G^\circ)_v$. Moreover, $G_v$ is Zariski closed, so $G_v / (G_v)^\circ$ is finite. Altogether, if $(G^\circ)_v$ is finite, then $(G_v)^\circ$ is finite and so is $G_v$.
\end{proof}


We end with the fact that, in a complex setting compatible with the real structures, the capacity of a real vector is independent of $\KK \in \{\RR, \CC\}$. This has interesting algorithmic implications: when approximating the capacity of a real vector it allows to use to use algorithms over $\CC$, e.g., as in \cite{GradflowArXiv}.

Let $G_\CC \subseteq \GL_N(\CC)$ be Zariski closed, self-adjoint and defined over $\RR$. Then $G_\RR := G_\CC \cap \GL_N(\RR)$ is Zariski closed and self-adjoint. Consider a rational representation $\pi \colon G_\CC \to \GL(V_\CC)$ defined over $\RR$. Then $\pi_\RR \colon G_\RR \to \GL(V_\RR)$ is a rational representation of $G_\RR$. Equip $V_\CC$ with a Hermitian inner product $\langle \cdot, \cdot \rangle$ on $V_\CC$ that is invariant under $K := G \cap \Un_N$ and compatible with $V_\RR$, i.e., $\langle v, w \rangle \in \RR$ for all $v,w \in V_\RR$. This puts us into the setting of \cite[§8]{RichardsonSlodowy}.

\begin{prop}[based on {\cite[Proposition~2.3]{SiagaPaper}}] \label{prop:RealVsComplexCapacity}
	\ \\
	Assume the setting above. Let $\capac_{G_\KK}(v)$ be the capacity of $v \in V_\KK$ under $G_{\KK}$ and let $\mathcal{N}_{\KK} = \{ v \in V_\KK \mid \capac_{G_\KK}(v) =0\}$ be the null cone under the action of $G_{\KK}$ on $V_\KK$.
	\begin{itemize}
		\item[(i)] For $v \in V_\RR$, we have the equality of capacities $\capac_{G_\RR}(v) = \capac_{G_\CC}(v)$. In particular, $\mathcal{N}_{\RR} = \mathcal{N}_{\CC} \cap V_\RR$.
		
		\item[(ii)] $\Ncal_{\RR} = V_\RR$ if and only if $\Ncal_{\CC} = V_\CC$.
	\end{itemize}
\end{prop}

\begin{proof}
	For part~(i), we have $\capac_{G_\RR}(v) \geq \capac_{G_\CC}(v)$ as $G_\RR \subseteq G_\CC$. Regarding the converse inequality, the capacity 
	$\capac_{G_\KK}(v)$ is attained at all elements of minimal norm in the closed orbit contained in $\overline{G_{\KK} \cdot v}$, by Kempf-Ness Theorem~\ref{thm:KempfNessAKRS}.
	Hence, we can reduce to studying a closed orbit $G_{\RR} \cdot v$.
	If $w$ is of minimal norm in $G_{\RR} \cdot v$, then it is of minimal norm in $G_{\CC} \cdot v$ by  \cite[Lemma~8.1]{RichardsonSlodowy}. Thus, $G_{\CC} \cdot w$ is closed by Kempf-Ness and hence $\|w\|^2 = \capac_{G_\CC}(v)$. This shows~(i).
	
	For part~(ii), $\Ncal_{\CC} = V_\CC$ directly implies $\Ncal_{\RR} = V_\RR$. Conversely, $V_\RR$ is Zariski dense in the irreducible complex variety $V_\CC$, so $\Ncal_\RR = V_\RR$ yields that $\Ncal_\CC$ contains the Zariski dense subset $\Ncal_\RR$. As $\Ncal_\CC$ is Zariski closed in $V_\CC$ (see Remark~\ref{rem:UsualStabilityGIT}), we must have $\Ncal_\CC = V_\CC$.
\end{proof}

%then todo: If $G \subseteq \GL_N(\RR)$ is Zariski closed and self-adjoint, then its Zariski closure $G_\CC$ in $\GL_N(\CC)$ is the algebraic group obtained by scalar extension. Thus, $G_{\CC} \cap \GL_N(\RR) = G$ and $\Lie(G_\CC) = \Lie(G) \oplus \imag \Lie(G)$. One has $G_{\CC} \subseteq \GL_N(\CC)$ is Zariski closed and self-adjoint \cite[Lemma~3.29]{Wallach}.
%Moreover, set $K := G \cap \Orth_N(\RR)$ and $U := G_{\CC} \cap \Un_N$. Then $K = U \cap \GL_N(\RR)$. Moreover, if we write $\Lie(G) = \Lie(K) \oplus \pfrak$ and $\Lie(G_{\CC}) = \Lie(U) \oplus \imag \Lie(U)$ as usual (Proposition~\ref{prop:LieAlgebraProperties}), then
%\[ \Lie(U) = \Lie(K) \oplus \imag \pfrak \qquad \text{and} \qquad
%\imag \Lie(U) = \imag \Lie(K) \oplus \pfrak \]
%(Do we need this ???)






%Moment Polytopes
\subsubsection{Moment Polytopes}

We explain how the moment maps induces so-called moment polytopes. They generalize weight polytopes, which arise in the case of torus actions. These polytopes be used to express the duality in \eqref{eq:KempfNessDuality}. Moreover, the combinatorics of these polytopes captures important complexity measures studied in Chapter~\ref{ch:BoundsMarginGap}.
In the latter we only work over $\CC$. Therefore, we restrict in the following to the complex numbers, and only comment on real moment polytopes in Remark~\ref{rem:RealMomentPolytopes}.

\bigskip

As a motivation of moment polytopes, we first describe how weight polytopes arise as images of the moment map.
For this, assume the Setting~\ref{set:MomentMap} for $\KK=\CC$, $G = T$ being a complex torus, and $\pi \colon T \to \GL(V)$ a rational representation with set of weights $\Omega(\pi)$. Remember that $V$ admits a weight space decomposition $V = \bigoplus_{\omega \in \Omega(\pi)} V_\omega$ and hence for $v \in V$ we have $v = \sum_\omega v_\omega$ for some $v_\omega \in V_\omega$. The weight polytope of $v$ is $\Delta_T(v) = \conv\{ \omega \mid v_\omega \neq 0 \}$.
We know from \eqref{eq:MomentMapGeneralTorus} in Example~\ref{ex:MomentMapTorus} that the moment map at $v$ is
	\[ \mu_T(v) = \sum_{\omega \in \Omega(\pi)} \frac{\|v_\omega\|^2}{\|v\|^2} \, \omega \, . \]
Moreover, we have seen in Example~\ref{ex:MomentMapTorus} that the weight spaces $V_\omega$ are pairwise orthogonal. Therefore, $\mu_T(v)$ is a convex combination of the weights and hence $\mu_T(v)$ lies in the \emph{relative} interior of $\Delta_T(v)$, i.e., $\mu_T(v) \in \relint(\Delta_T(v))$. In fact, it was proven independently by Atiyah \cite[Theorem~2]{AtiyahConvexity} and by Guillemin-Sternberg \cite[Theorem~4]{GuilleminSternberg} that
\begin{equation}\label{eq:WeightPolytopeVsMomentMap} %formerly eq:PolytopeVsMomentMap
	\mathrm{relint}\, P_v(A) = \mu(\GT_d \cdot v) \qquad \text{and so} \qquad
	\Delta_{T}(v) = \overline{ \lbrace \mu_{T}(t \cdot v) \mid t \in T \rbrace } \, .
\end{equation}
The statements in \cite{AtiyahConvexity,GuilleminSternberg} rather apply to a projectivized setting. We provide a brief translation for readers that are unfamiliar with these topics.

\begin{remark}[based on {\cite[Remark~B.1]{DiscretePaper}}] \label{rem:ProjectiveSetting}
	Remember that the moment map $\mu_T \colon V \backslash \{0\} \to \pfrak = \imag \Lie(T_K)$ is invariant under non-zero scalars and therefore factors through the projective space $\PP(V)$ via a map $\bar{\mu} \colon \PP(V) \to \pfrak$.
	For a non-zero $v \in V$, let $[v]$ be the point in $\PP(V)$ that represents the line $\CC v$. Then $T$ naturally acts on $\PP(V)$ and $\bar{\mu}$ is the moment map for this action. This action fits the setting of \cite{AtiyahConvexity,GuilleminSternberg}, because $\PP(V)$ is a compact K\"ahler manifold.
	
	The results \cite[Theorem~2]{AtiyahConvexity} and \cite[Theorem~4]{GuilleminSternberg} give
		\[ \Delta_T(v) = \bar{\mu} \left( \overline{T \cdot [v]} \right).\]
	For~\eqref{eq:WeightPolytopeVsMomentMap}, we need a statement for
	the orbit of $v$ rather than the orbit closure of $[v]$. The closure $\overline{T\cdot [v]}$ is the disjoint union of finitely many $T$ orbits. The orbits relate to $\Delta_T(v)$ as follows.  For each open face $F$ of $\Delta_T(v)$ the set $\bar{\mu}^{-1}(F) \cap \overline{T \cdot [v]}$ is a single $T$-orbit in $\PP(V)$, \cite[Theorem~2]{AtiyahConvexity}. In particular, for $F = \relint \Delta_T(v)$ we obtain the orbit $T \cdot [v]$. This yields~\eqref{eq:WeightPolytopeVsMomentMap}, since $\bar{\mu}(T \cdot [v]) = \mu(T \cdot v)$.
	\hfill\remSymbol
\end{remark}

We point out how Equation~\eqref{eq:WeightPolytopeVsMomentMap} connects Hilbert-Mumford and Kempf-Ness for torus actions. By Hilbert-Mumford Theorem~\ref{thm:HMtorusWeightPolytope}(c) polystability is equivalent to $0 \in \relint \Delta_T(v)$, which translates with \eqref{eq:WeightPolytopeVsMomentMap} to $0 \in \mu(T \cdot v)$. The latter is equivalent to the statement for polystability in Kempf-Ness, Theorem~\ref{thm:KempfNessAKRS}(e).

\bigskip

The fact that the image of the moment map yields a polytope remarkably generalizes to the non-commutative setting, giving so-called \emph{moment polytopes}. We need the latter only in the case $G = \SL_m(\CC)^d$. Thus for concreteness, assume the Setting~\ref{set:MomentMap} for $G = \SL_m(\CC)^d$ and corresponding moment map $\mu_G$.
Then for fixed $v \in V \setminus \{0\}$, the set $\{\mu_G(g \cdot v) \mid g \in G\}$ gives rise to a polytope as follows. 

Let $\spec \colon \Sym_m(\CC) \to \RR^m$ be the function sending a Hermitian matrix to its eigenvalues in decreasing order. Recalling that $\imag \Lie(K) \subseteq \Sym_m(\CC)^d$ is block-diagonally embedded in $\CC^{dm \times dm}$, we set
	\begin{align*}
		s \colon \imag \Lie(K) \to \left( \RR^m \right)^d, \quad \diag(X_1, \ldots, X_d) \mapsto \big( \spec(X_1), \ldots, \spec(X_d) \big).
	\end{align*}
Then for $v \in V \setminus \{0\}$ the set
	\begin{equation}\label{eq:defnMomentPolytope}
		\Delta_G(v) := \overline{ \left\lbrace s \big( \mu_G(w) \big) \mid w \in G \cdot v  \right\rbrace }
	\end{equation}
is a convex polytope with rational vertices, see \cite{brion1987sur}, \cite{GuilleminSternberg}, \cite{kirwan1984convexity} or \cite[Appendix]{NessStratification} by Mumford. We call $\Delta_G(v)$ the \emph{moment polytope}\index{moment polytope} of $v$. Noting that $\| X \|_F = \| \spec(X) \|_2$ for any $X \in \Sym_m(\CC)$ we have $\| \mu_G(v) \|_F = \| s(\mu_G(v)) \|_2$ for all $v \in V \backslash \{0\}$. Thus, we can formulate the duality from Equation~\eqref{eq:KempfNessDuality} also as follows:
	\begin{equation}\label{eq:MomentPolytopeVsCapacity}
		\capac_G(v) = 0 \qquad \Leftrightarrow \qquad  \dist \big( 0, \Delta_G(v) \big) > 0 
		\qquad \Leftrightarrow \qquad 0 \notin \Delta_G(v),
	\end{equation}
This will motivate the definition of two precision parameters in Definition~\ref{defn:WeightMarginGapConstant}. Let us briefly comment how to define $\Delta_G(v)$ for an arbitrary group $G$.

\begin{remark}[General Definition of $\Delta_G(v)$]
	For a general group $G$ as in Setting~\ref{set:MomentMap}, one can fix a fundamental Weight chamber\footnote{It is also called positive Weyl chamber. In our concrete setting $G \subseteq \GL_N(\CC)$, a natural choice is to take the fundamental Weyl chamber with respect to the group $G \cap \Bor_N(\CC)$ of upper triangular matrices in $G$.}
	$C(G) \subseteq \imag \Lie(T_K) \subseteq \RR^N$, see \cite[Definition~8.20]{HallBook} or \cite[Definition~3.1.11]{GoodmanWallachBook}. For any $X \in \pfrak = \imag \Lie(K)$, this chamber $C(G)$ intersects the $\Ad(K)$-orbit $\{ k X k\HT \mid k \in K \}$ in a single point, denoted $s(X)$. This yields the moment polytope $\Delta_G(v)$, defined exactly as in \eqref{eq:defnMomentPolytope}.
	Note that for any $X \in \pfrak$ there is some $k \in K$ with $s(X) = k X k\HT$, and so $\|s(X)\| = \|k X k\HT\| = \| X \|$ by unitary invariance of the Frobenius norm. Thus, Equation~\eqref{eq:MomentPolytopeVsCapacity} holds in general.
	
	If $G = \SL_m(\CC)$, then the positive Weyl chamber is
		\[ C(G) = \{ \diag(x) \mid x \in \RR^m, \, x_+ = 0, \, x_1 \geq x_2 \geq \cdots \geq x_m\} \subseteq \imag \Lie(T_K) \cong \onePerp \, . \]
	For $X \in \imag \Lie(\SU_m)$ we indeed have $\{ k X k\HT \mid k \in \SU_m \} \cap C(G) = \{\spec(X)\}$.
	\hfill\remSymbol
\end{remark} %perhaps \cite[Proposition~3.1.20]{GoodmanWallachBook} is useful

We end by giving references for moment polytopes in the real case. 

\begin{remark}[Moment Polytopes for $\KK = \RR$] \label{rem:RealMomentPolytopes}
	Interestingly, one can as well consider moment polytopes over the reals, which can then be described as sub-polytopes of complex moment polytopes \cite[Theorem~3.1]{osheaSjamaar2000moment}. Recent studies on the facets of these real moment polytopes can be found in the preprint \cite{paradan2020moment}. We refer to \cite{osheaSjamaar2000moment, paradan2020moment} and the literature therein for further information on real moment polytopes.
	\hfill\remSymbol
\end{remark}

Since all actions studied in Chapter~\ref{ch:BoundsMarginGap} are defined over $\RR$ and allow for moment polytopes over $\RR$, Remark~\ref{rem:RealMomentPolytopes} naturally leads to the following question.\footnote{The author only recently became aware of the concept of moment polytopes for $\KK = \RR$.}

\begin{question}
	Do the upper bounds from Chapter~\ref{ch:BoundsMarginGap} for the gap via complex moment polytopes
	also hold for a gap defined analogously via real moment polytopes?
\end{question} %todo perhaps move to Chapter~4







%=========== King's Criterion ========================

\section{King's Criterion for Quivers}\label{sec:King}

This section is based on \cite[Appendix~A]{SiagaPaper}. Its aim is to characterize stable elements under the left-right action when $\KK = \CC$. For this, we can use results from \cite{King} on stability of quiver representations.
The main result is the following.


\begin{theorem}\label{thm:KingStability}	
	Consider the left-right action of $H := \SL_{m_1}(\CC) \times \SL_{m_2}(\CC)$ on $V:= (\CC^{m_1 \times m_2})^n$.
	Then $Y = (Y_1,\ldots,Y_n) \in V$ is stable under $H$ if and only if 
	\begin{itemize}
		\item[(i)] the matrix $(Y_1 | \ldots | Y_n) \in \CC^{m_1 \times n m_2}$ has rank $m_1$, and
		\item[(ii)] for all subspaces  $V_1 \subseteq \CC^{m_1}$, $\lbrace 0 \rbrace \subsetneq V_2 \subsetneq \CC^{m_2}$ that satisfy $Y_i V_2 \subseteq V_1$ for all $i \in [n]$, one has $m_2 \dim V_1 > m_1 \dim V_2$.
	\end{itemize}
\end{theorem}

In the following, we explain how to deduce Theorem~\ref{thm:KingStability} from \cite{King}. For concreteness, we directly restrict the general setting in \cite{King} to the quiver of interest.
Let $Q$ be the $n$-Kronecker quiver\index{Kronecker quiver} with two vertices and $n$ arrows:
\begin{center}
	\begin{tikzcd}
		1  & 2 \ar[l, shift left = 4pt, bend left] \ar[l, draw=none, "\raisebox{+0.7ex}{\vdots}" description] \ar[l, bend right, shift right = 3pt]
	\end{tikzcd}
\end{center}
Recall from Example~\ref{ex:QuiverRep} that given a dimension vector $\alpha = (m_1,m_2)$ the groups $G := \GL_\alpha(\CC) = \GL_{m_1}(\CC) \times \GL_{m_2}(\CC)$ and $H := \SL_\alpha(\CC) = \SL_{m_1}(\CC) \times \SL_{m_2}(\CC)$ act on $V = \Rscr(Q,\alpha) \cong (\CC^{m_1} \times \CC^{m_2})^n$ via
	\[ (g_1,g_2) \cdot (Y_1,\ldots,Y_n) = (g_1 Y_1 g_2^{-1}, \ldots, g_1 Y_n g_2^{-1}) \, . \]
After precomposition with the automorphism $(g_1,g_2) \mapsto (g_1, g_2^{-\mathsf{T}})$ this is the left-right action of $G$ (respectively $H$) on $V$, compare Example~\ref{ex:RepLeftRight}. Thus, $Y \in V$ is semi/poly/stable under the $H$-Kronecker quiver action if and only if it is semi/poly/stable under the $H$-left-right action. Hence, we can deduce Theorem~\ref{thm:KingStability} by considering the Kronecker quiver action.

For this, we need another action of $G = \GL_\alpha(\CC)$ from \cite{King}. Let $\chi_\theta$ be the character of $G$ given by $\theta := (m_2, -m_1)$, i.e.,
	$ \chi_{\theta}(g_1,g_2) = \det(g_1)^{m_2} \det(g_2)^{-m_1} .$
We consider the action of $G$ on $V \times \CC$, where $G$ acts on $V$ by the Kronecker quiver action and on $\CC$ by the character $\chi_\theta^{-1}$, i.e.,
\begin{equation}\label{eq:KingChiThetaAction}
	g \cdot (X,z) := (g \cdot X, \chi_{\theta}^{-1}(g)z), \quad\text{ where } \quad
	\chi_{\theta}^{-1}(g) = \det(g_1)^{-m_2} \det(g_2)^{m_1} .
\end{equation}
Given $Y \in V$, we usually consider this action for $\hat{Y} := (Y,1)$.
Note that $\langle \theta, \alpha \rangle = 0$; an important assumption in \cite{King} which ensures that the central subgroup
	\[ \Delta := \big\{ (t \Id_{m_1}, t \Id_{m_2}) \mid t \in \CC^\times \big\} \subseteq G \]
is always contained in the stabilizer $G_{\hat{Y}}$. In \cite[Definition~2.1]{King} defines $\chi_{\theta}$-(semi)stability for $Y$. For us, the following characterizations are important.\footnote{The reader may regard these characterizations as a definition of $\chi_{\theta}$-(semi)stable.}

\begin{lemma}[{\cite[Lemma~2.2]{King}}] \label{lem:KingLemma-2-2}
	Let $Y \in V = (\CC^{m_1 \times m_2})^n = \Rscr (Q, \alpha)$ and set $\hat{Y} := (Y,1) \in V \times \CC$. Then
	\begin{itemize}
		\item[(a)] $Y$ is $\chi_\theta$-semistable if and only if $(V \times \{0\}) \cap \overline{G \cdot \hat{Y}} = \emptyset$.
		
		\item[(b)] $Y$ is $\chi_{\theta}$-stable if and only if $G \cdot \hat{Y}$ is closed and $G_{\hat{Y}} / \Delta$ is finite.\footnote{King works with the Zariski topology, while we apply this result with respect to the Euclidean topology. Thus, we use Corollary~\ref{cor:ClosureComplexCase} here.}
	\end{itemize}
\end{lemma}

To prove Theorem~\ref{thm:KingStability}, we will later show that $Y$ is $\chi_\theta$-stable if and only if it is $H$-stable. The items~(i) and~(ii) from Theorem~\ref{thm:KingStability} stem from the following stability notions.

\begin{defn}[{\cite[Definition~1.1]{King}}] \label{defn:KingThetaStability}
	Let $Y \in V = (\CC^{m_1 \times m_2})^n = \Rscr(Q,\alpha)$. We write $(\CC^{m_1}, \CC^{m_2}; Y)$ if we want to stress that we view $Y$ as a representation of the Kronecker quiver (Definition~\ref{defn:QuiverRepresentation}).
	We say $Y$ is \emph{$\theta$-semistable}\index{thetasemistable@$\theta$-semistable} if
	for all quiver-subrepresentations of $(\CC^{m_1}, \CC^{m_2}; Y)$, i.e., all subspaces $V_1 \subseteq \CC^{m_1}$, $V_2 \subseteq \CC^  {m_2}$ with $Y_i V_2 \subseteq V_1$ for all $i$, we have
	\begin{equation}\label{eq:thetaStable}
		\langle \theta, (\dim V_1, \dim V_2) \rangle = m_2 \dim V_1 - m_1 \dim V_2 \geq 0.
	\end{equation}
	$Y$ is \emph{$\theta$-stable}\index{thetastable@$\theta$-stable} if the inequality in~\eqref{eq:thetaStable} is strict for all non-zero proper subrepresentations. Here, non-zero means $V_1 \neq 0$ or $V_2 \neq 0$, while proper means $V_1 \subsetneq \CC^{m_1}$ or $V_2 \subsetneq \CC^{m_2}$. 
	\hfill\defnSymbol
\end{defn}

The concepts of $\theta$-(semi)stability and $\chi_\theta$-(semi)stability agree.

\begin{prop}[{\cite[Proposition~3.1]{King}}] \label{prop:KingProposition-3-1}
	Let $Y \in V = (\CC^{m_1 \times m_2})^n$. Then $Y$ is $\chi_{\theta}$-semistable (respectively $\chi_{\theta}$-stable) if and only if $Y$ is $\theta$-semistable (respectively $\theta$-stable). 
\end{prop}

To show that $Y$ is $\chi_\theta$-stable if and only if it is $H$-stable, we provide a lemma. 

\begin{lemma}[{\cite[Lemma~A.1]{SiagaPaper}}]
	\label{lem:RelationToKing}
	Let $Y \in V = (\CC^{m_1 \times m_2})^n$ and $z \in \CC^\times$, and set $\hat{Y} := (Y,1) \in V \times \CC$. Fix an $(m_1 m_2)$-root function on $\CC$. Then
	\begin{itemize}\itemsep 3pt
		\item[(a)] $(X,z) \in G \cdot \hat{Y} \quad \Leftrightarrow \quad z^{\frac{1}{m_1 m_2}} X \in H \cdot Y$
		\item[(b)] $(X,z) \in \overline{G \cdot \hat{Y}} \quad \Leftrightarrow \quad z^{\frac{1}{m_1 m_2}} X \in \overline{H \cdot Y}$
		\item[(c)] $\left( \exists \, X \in V  \colon (X,0) \in \overline{G \cdot \hat{Y}} \right) \quad \Leftrightarrow \quad 0 \in \overline{H \cdot Y}$. 
		\item[(d)] The stabilizer $H_Y$ is finite if and only if $G_{\hat{Y}} / \Delta$ is finite.\footnote{This part is not included in \cite[Lemma~A.1]{SiagaPaper}, but appeared later in \cite[Appendix~A]{SiagaPaper}. We note that the argument in \cite{SiagaPaper} contains a mistake. In particular, it states $H_Y \cong G_{\hat{Y}} / \Delta$, which is in general not true. We provide a correction.}
	\end{itemize}
\end{lemma}

\begin{proof}
	To prove (a), take $g \in G$ with $(X,z) = g \cdot \hat{Y}$. By Equation~\eqref{eq:KingChiThetaAction}, we have $g \cdot Y = X$ and $\det(g_1)^{-m_2} \det(g_2)^{m_1} = z$. The latter shows that there exist some roots\footnote{Note that in general not all choices of roots will work, but there always exists a certain choice with the desired properties.} $\det(g_1)^{-\frac{1}{m_1}}, \det(g_2)^{-\frac{1}{m_2}} \in \CC^\times$ such that 
		\[ \det(g_1)^{-\frac{1}{m_1}} \det(g_2)^{\frac{1}{m_2}} = z^{\frac{1}{m_1 m_2}}, \; \text{ i.e., } \; h := \big( \det(g_1)^{-\frac{1}{m_1}} g_1, \, \det(g_2)^{-\frac{1}{m_2}} g_2 \big) \in H \]
	satisfies $h \cdot Y = z^{\frac{1}{m_1 m_2}} X$. 
	Conversely, given the latter for some $h = (h_1, h_2) \in H$, we define $g := \big( z^{-\frac{1}{m_1 m_2}} h_1, \, h_2 \big)$ and compute $g \cdot \hat{Y} = (X,z)$ using~\eqref{eq:KingChiThetaAction}.
	
	Part~(b) follows from applying part~(a) to a sequence in the respective orbit that tends to a point in the orbit closure.
	
	For part~(c), note that if $Y=0$ then $(0,0) \in \overline{G \cdot \hat{Y}}$ and $0 \in \overline{H \cdot Y}$. It remains to consider $Y \neq 0$. Take $X \in V$ and let $g^{(k)} \in G$ be a sequence such that $g^{(k)} \cdot \hat{Y}$ tends to $(X,0)$ as $k \to \infty$. 
	Since $\chi_{\theta}^{-1}(g^{(k)}) \neq 0$ for all $k$, we apply (a) to obtain $
	Y_k := \left[ \chi_{\theta}^{-1}(g^{(k)}) \right]^{\frac{1}{m_1 m_2}} g^{(k)} \cdot Y \in H \cdot Y $
	for all~$k$. With $g^{(k)} \cdot \hat{Y} \to (X,0)$ for $k \to \infty$ we conclude that the sequence $Y_k$ tends to $0 \in V$. On the other hand, assume there exist $Y_k \in H \cdot Y$ with $Y_k \to 0$ as $k \to \infty$. Since $Y \neq 0$, we have $Y_k \neq 0$ and hence $c_k := \| Y_k \|^{\frac{m_1 m_2}{2}} \neq 0$ for all $k$. Thus, setting 
	$X_k := c_k^{-\frac{1}{m_1 m_2}} Y_k$
	and applying part~(a) to $Y_k = c_k^{\frac{1}{m_1m_2}} X_k$ gives $(X_k, c_k) \in G \cdot \hat{Y}$. The latter sequence tends to $(0,0) \in V \times \CC$, noting that $\|X_k\| = \|Y_k\|^{\frac{1}{2}}$ by the choice of $c_k$.
	
	For part~(d), first note that any $h = (h_1,h_2) \in H_Y$ stabilizes $\hat{Y}$ under the action~\eqref{eq:KingChiThetaAction}, because $h_1$ and $h_2$ have determinant one. Therefore, we have a group morphism 
		\[ \varphi \colon H_Y \to G_{\hat{Y}} / \Delta , \quad (h_1, h_2) \mapsto \overline{(h_1, h_2)} \, .\]
	Its kernel is $H_Y \cap \Delta = \{ (t \Id_{m_1}, t \Id_{m_2}) \mid t \in \CC^\times, \, t^{m_1} = t^{m_2} = 1 \}$, which is finite. Moreover, $\varphi$ is surjective by the following. If $g = (g_1,g_2) \in G_{\hat{Y}}$ then $\chi_\theta^{-1}(g) \cdot 1 = 1$ translates to $\det(g_2)^{m_1} = \det(g_1)^{m_2} =: \lambda$.
	Take an $(m_1m_2)$-root to obtain
		\[ t := \lambda^{- \frac{1}{m_1 m_2}} = \det(g_1)^{-\frac{1}{m_1}}  = \det(g_2)^{-\frac{1}{m_2}} .\]
	%Take an $(m_1 m_2)$-root of $\lambda^{-1}$, i.e., choose $t \in \CC^\times$ such that $t^{m_1 m_2} = \lambda^{-1}$. Note that $t^{m_k} = \det(g_k)^{-1}$ for $k=1,2$.
	Then $h := (t g_1, t g_2) \in H$, but $h$ also stabilizes $Y$ as $g \in G_{\hat{Y}}$, so $h \in H_Y$. By construction, $\varphi(h) = \overline{g} \in G_{\hat{Y}} / \Delta$, hence $\varphi$ is surjective.
	
	Altogether, $H_Y / \ker(\varphi) \cong G_{\hat{Y}} / \Delta$. Since $\ker(\varphi)$ is finite, we deduce that $H_Y$ is finite if and only if $G_{\hat{Y}} / \Delta$ is finite.
\end{proof}

%finally prove the theorem; and state the remark(?)

With the help of Lemma~\ref{lem:RelationToKing} we finally prove Theorem~\ref{thm:KingStability}.

\begin{proof}[Proof of Theorem~\ref{thm:KingStability}]
	By Proposition~\ref{prop:KingProposition-3-1}, the matrix tuple $Y = (Y_1,\ldots,Y_n)$ is $\chi_\theta$-stable if and only if it is $\theta$-stable. First, we show that the former is equivalent to being $H$-stable under the Kronecker quiver action. Then we rephrase $\theta$-stability as the (shrunk subspace) conditions~(i) and~(ii).
	
	Let $G_{\hat{Y}}$ denote the $G$-stabilizer of $\hat{Y} = (Y,1)$ under the action~\eqref{eq:KingChiThetaAction}. By Lemma~\ref{lem:KingLemma-2-2}, $Y$ is $\chi_\theta$-stable if and only if the orbit $G \cdot \hat{Y}$ is closed and the group $G_{\hat{Y}}/\Delta$ is finite. The group $G_{\hat{Y}} / \Delta$ is finite if and only if $H_Y$ is finite, by Lemma~\ref{lem:RelationToKing}(d). For $Y=0$, we have $H_Y = H$, which is not finite.
	
	Thus, it remains to show for $Y \neq 0$ that $G \cdot \hat{Y}$ is closed if and only if $H \cdot Y$ is closed. If $G \cdot \hat{Y}$ is closed and $X \in \overline{H \cdot Y}$, then $(X,1) \in \overline{G \cdot \hat{Y}} = G \cdot \hat{Y}$ using Lemma~\ref{lem:RelationToKing}(b), and hence $X \in H \cdot Y$ by Lemma~\ref{lem:RelationToKing}(a). Conversely, if $H \cdot Y$ is closed with $Y \neq 0$ then $0 \notin \overline{H \cdot Y}$. Thus, Lemma~\ref{lem:RelationToKing}(c) yields $\overline{G \cdot \hat{Y}} \cap \big( V \times \lbrace 0 \rbrace \big) = \emptyset$. Hence, any $(X,z) \in \overline{G \cdot \hat{Y}}$ must satisfy $z \in \CC^\times$, so $z^{\frac{1}{m_1m_2}} \in \overline{H \cdot Y} = H \cdot Y$ by Lemma~\ref{lem:RelationToKing}(b). We conclude $(X,z) \in G \cdot \hat{Y}$ using Lemma~\ref{lem:RelationToKing}(a).
	
	For $Y$ being $\theta$-stable, recall from Definition~\ref{defn:KingThetaStability} that for all non-zero proper quiver-subrepresentations of $(\CC^{m_1}, \CC^{m_2}; Y)$ the inequality~\eqref{eq:thetaStable} has to be strict:
	\begin{equation*}
		\langle \theta, (\dim V_1, \dim V_2) \rangle = m_2 \dim V_1 - m_1 \dim V_2 > 0.
	\end{equation*}
	Here, non-zero means $V_1 \neq 0$ or $V_2 \neq 0$, while proper means $V_1 \subsetneq \CC^{m_1}$ or $V_2 \subsetneq \CC^{m_2}$. Since $V_1 \neq 0$ and $V_2 = 0$ gives strict inequality in \eqref{eq:thetaStable}, it is enough to consider $V_2 \neq 0$. On the other hand, strict inequality in \eqref{eq:thetaStable} holds for all proper subrepresentations satisfying $V_1 \subsetneq \CC^{m_1}$ and $V_2 = \CC^{m_2}$ if and only if there is \emph{no} proper subrepresentation of this form, i.e., if and only if $\mathrm{rank}(Y_1,\ldots,Y_n) = m_1$. Hence, by requiring the latter condition we can restrict to the case $V_2 \subsetneq \CC^{m_2}$. Altogether, we rephrased the $\theta$-stability of $Y$ as (i) and (ii) in the statement.
\end{proof}

Similarly, we obtain a characterization for being semistable under the left-right action of $H$. The statement was proven differently in \cite[Proposition~2.1]{BurginDraisma}, and we revisit it in Theorem~\ref{thm:nullconeLeftRight}.

\begin{prop}\label{prop:KingSemistable}
	Consider the left-right action of $H := \SL_{m_1}(\CC) \times \SL_{m_2}(\CC)$ on $V:= (\CC^{m_1 \times m_2})^n$.
	Then $Y = (Y_1,\ldots,Y_n) \in V$ is semistable under $H$ if and only if
	for all subspaces  $V_1 \subseteq \CC^{m_1}$, $V_2 \subseteq \CC^{m_2}$ that satisfy $Y_i V_2 \subseteq V_1$ for all $i \in [n]$, one has $m_2 \dim V_1 \geq m_1 \dim V_2$.
\end{prop}

\begin{proof}
	This is based on \cite[Remark~A.2]{SiagaPaper}. Remember $Y$ is $H$-semistable under the left-right action if and only if it is $H$-semistable under the Kronecker quiver action.
	By Lemma~\ref{lem:RelationToKing}(c), the latter is equivalent to 
	\begin{equation*}
		\big( V \times \lbrace 0 \rbrace \big) \cap \overline{G \cdot \hat{Y}} \neq \emptyset,
	\end{equation*}
	which in turn is equivalent to $Y$ being $\chi_\theta$-semistable, see Lemma~\ref{lem:KingLemma-2-2}. By Proposition~\ref{prop:KingProposition-3-1}, $\chi_\theta$-semistability is equivalent to $\theta$-semistability, and that translates via Definition~\ref{defn:KingThetaStability} to the desired conditions.
\end{proof}

%\begin{remark}[{\cite[Remark~A.2]{SiagaPaper}}]
%	\label{rem:kingSemistable}
%	Proposition~3.1 in \cite{King} provides an alternative proof of the complex analogue of 
%	It  states that $Y$ is $\chi_\theta$-semistable if and only if $(\CC^{m_1}, \CC^{m_2}; Y)$ is $\theta$-semistable. The former holds if and only if 
%	\begin{equation*}
%		\big( V \times \lbrace 0 \rbrace \big) \cap \overline{G \cdot \hat{Y}} \neq \emptyset,
%	\end{equation*}
%	i.e., if and only if $Y$ is semistable under the action of $H$, by Lemma~\ref{lem:RelationToKing}(c).
%	On the other hand, the proof of Theorem~\ref{thm:KingStability} shows that $(\CC^{m_1}, \CC^{m_2}; Y)$ is $\theta$-semistable if and only if \eqref{eq:thetaStable} holds for all subspaces $V_1 \subseteq \CC^{m_1}$, $V_2 \subseteq \CC^{m_2}$ satisfying $Y_i V_2 \subseteq V_1$ for all $i = 1,\ldots,n$.
%	\hfill\remSymbol
%\end{remark}







%=========== Popov's Criterion ========================

\section{Popov's Criterion for solvable Groups} \label{sec:Popov}

In this subsection we present Popov's Criterion for Zariski closed orbits under a connected solvable group. First, we briefly state the criterion in its general form. Afterwards, we specialize it to the very concrete setting in which we will apply it later. Since the criterion requires an algebraically closed field, we end with a lemma that allows to deduce polystability over $\RR$, given the complex orbit is closed, and the rational representation is defined over $\RR$.

Let $G$ be a connected solvable group over $\CC$. Then $G$ is the semi-direct product of its unipotent radical $U$ and a maximal torus $T$, see Theorem~\ref{thm:LeviDecomposition}. By Proposition~\ref{prop:UnipotentCharacter}, $\Xfrak(U) = 0$ and hence the character group $\Xfrak(G)$ of $G$ can be identified with $\Xfrak(T)$ via restriction. We use this identification to view $\Xfrak(T)$ as a subset of the coordinate ring $\CC[G]$. Assume $G$ acts algebraically on an affine variety $Z$. For $z \in Z$, consider the orbit map $\nu_{G \cdot z} \colon G \to Z, \; g \mapsto g \cdot z$ and its pullback map $\nu_{G \cdot z}^\ast \colon \CC[Z] \to \CC[G]$. Then $R_z := \nu_{G \cdot z}^\ast(\CC[Z])$ is a subalgebra of $\CC[G]$. Therefore, $\Xfrak_{G \cdot z} := \{ \chi \in \Xfrak(T) \mid \chi \in R_z \}$ is a semigroup, where we identified $\Xfrak(T) \subseteq \CC[G]$.

\begin{theorem}[{Popov's Criterion, \cite[Theorem~4]{popov1989closed}}] \label{thm:PopovCriterion}
	\ \\
	Assume $G$ and $Z$ are as above, and let $z \in Z$. The orbit $G \cdot z$ is Zariski closed in $Z$ if and only if the semigroup $\Xfrak_{G \cdot z}$ is a group.
\end{theorem}

%(TODO deal with connected issue here, or later in RDAGs)   %todo adjust to include non-connected, i.e., $T$ is allowed to be a diagonalizable group
%general idea: \Xfrak_{G \cdot Y} = \Xfrak_{G^\circ \cdot Y} \times Torsion, Torsion may be zero, even if not, then no problem as torsion semigroup is automatically a group. Hence, \Xfrak_{G \cdot Y} is a group iff \Xfrak_{G^\circ \cdot Y} is a group. Clear, $G \cdot z$ is Zar closed if $G^\circ \cdot z$ is. Other direction clear for upper triangular, how about general solvable?? -> should also be true as we have semidirect product T \cdot U

\begin{remark}\label{rem:PopovCriterion}
	The Criterion contains the following special cases.
	\begin{itemize}
		\item[(i)] If $G = U$ is unipotent, then $\Xfrak(G)$ is trivial, compare Proposition~\ref{prop:UnipotentCharacter}. Hence, $\Xfrak_{G \cdot z}$ is the trivial group for any $z \in Z$ and therefore all orbits $G \cdot z$ are Zariski closed, compare \cite[Corollary~3]{popov1989closed}. Thus, Popov's Criterion specializes for unipotent groups to the Kostant-Rosenlicht Theorem, see \cite[Theorem~2]{rosenlicht1961onQuotient}.
		
		\item[(ii)] If $G = T$ is a torus, then Popov's Criterion specializes to \cite[Corollary~4]{popov1989closed}. If $Z=V$ is a rational representation of $G = T$, then this gives a reformulation of Theorem~\ref{thm:HMtorusWeightPolytope}(c),
		the Hilbert Mumford Criterion via the Newton polytope.
		\hfill\remSymbol
	\end{itemize}
\end{remark}

Now, we specialize to the setting in which we will apply Popov's Criterion. The following is similar to \cite[Section~9]{popov1989closed}. Let $G \subseteq \GL_m(\CC)$ be a subgroup consisting of upper triangular matrices, which acts on $(\CC^{m})^n \cong \CC^{m \times n}$ by left multiplication. Then $G$ is a semi-direct product of $T$, the group of diagonal matrices in $G$, and $U$, the group of unipotent upper triangular matrices in $G$.

The following notation is unusual from the perspective of algebraic geometry.\footnote{Usually, small letters denote constants and capital letters denote coordinate functions. Here, it is the other way around.} We adjusted to the notation of Sections~\ref{sec:TDAGs} and~\ref{sec:RDAGsGaussianGroupModels} for an easy comparison.
Denote the coordinate functions on $G$ by $x_{i,j} \in \CC[G]$,  $i,j \in [m]$, and those of $\CC^{m \times n}$ by $f_{i,l} \in \CC[\CC^{m \times n}]$, $i \in [m], \, l \in [n]$. For matrix $Y \in \CC^{m \times n}$, the pullback of the orbit map $\nu_{G \cdot Y}$ is given by
\[ \nu_{G \cdot Y}^*(f_{i,l}) = \sum_{j=1}^m x_{i,j} Y_{j,l} \]
and therefore
\begin{equation}\label{eq:PopovRY}
	R_Y = \nu_{G \cdot Y}^* \big( \CC[\CC^{m \times n}] \big) = \CC \Big[ \sum_{j=1}^m Y_{j,l} x_{i,j} \mid i \in [m], l \in [n] \Big] \subseteq \CC[G].
\end{equation} %todo: perhaps more explanations about R_Y and the semigroup
Since $T \subseteq \GT_m(\CC)$, we have a surjection $\varphi \colon \Xfrak(\GT_m(\CC)) \cong \ZZ^m \twoheadrightarrow \Xfrak(T)$ of abelian groups, see Proposition~\ref{prop:Characters}. Therefore, $\Xfrak(T) \cong \ZZ^m / \ker(\varphi)$ and we may write
\begin{align*}
	\Xfrak_{G \cdot Y} = \left\lbrace (d_1,\ldots,d_m) \in \Xfrak(T) \mid x_{11}^{d_1} \cdots x_{mm}^{d_m} \in R_Y \right\rbrace 
\end{align*}
using this identification.

\medskip

We finish the section with an argument how to deduce polystability over the reals if the representation is defined over $\RR$.
As mentioned in the proofs of \cite[Corollary~5.3]{birkes1971orbits} and \cite[Proposition~2.21]{DM21MatrixNormal} the next statement follows from \cite[Proposition~2.3]{BorelHarishChandra}.
We stress that the group does \emph{not} have to be reductive.

\begin{lemma}\label{lem:PopovForReal}
	Let $G$ be a connected complex algebraic group and $\pi \colon G \to \GL(V)$ be a rational representation, both defined over $\RR$. Let $v \in V_\RR$ and suppose that $G \cdot v$ is Euclidean closed in $V$. Then $G_{\RR} \cdot v$ is Euclidean closed in $V_\RR$.
\end{lemma}

\begin{proof} %todo point out connected is assumed in Borel HarishChandra
	By \cite[Proposition~2.3]{BorelHarishChandra} (Proposition~\ref{prop:BorelHarishChandraProp2-3}), $(G \cdot v) \cap V_\RR$ is\footnote{In general, $(G \cdot v) \cap V_\RR$ and $G_\RR \cdot v$ do not have to be equal (see Remark~\ref{rem:RealOrbits}), but the latter is contained in the former}
	a finite union of Euclidean closed $(G_{\RR})^\circ$-orbits, where $(G_{\RR})^\circ$ denotes the Euclidean identity component. One of these closed orbits must be $(G_\RR)^\circ \cdot v$. As $G_\RR$ is a real algebraic variety it has finitely many Euclidean-connected components by Theorem~\ref{thm:Whitney}.
	Choose representatives $g_1, \ldots, g_k$ of $G_{\RR} / (G_\RR)^\circ$. Since $G_\RR$ is a Lie group, the multiplication with $g_i$ is a homeomorphism and we conclude that
		\[ G_\RR \cdot v = \bigcup_{i=1}^k g_i \, \big( (G_\RR)^\circ \cdot v \big)\]
	is Euclidean closed as a finite union of Euclidean closed sets.
\end{proof}



















